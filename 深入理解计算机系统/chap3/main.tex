\documentclass[11pt]{article}

\usepackage[fleqn]{amsmath}
\usepackage{amssymb}
\usepackage[english]{babel}
\usepackage{bookmark}
\usepackage{booktabs}
\usepackage{capt-of}
\usepackage{colortbl}
\usepackage{dcolumn}
\usepackage{fancyhdr}
\usepackage[T1]{fontenc}
\usepackage{graphicx}
\usepackage{grffile}
\usepackage{hyperref}
\usepackage[utf8]{inputenc}
\usepackage{listings}
\usepackage{longtable}
\usepackage{rotating}
\usepackage{textcomp}
\usepackage{tikz}
\usepackage[normalem]{ulem}
\usepackage{verbatim}
\usepackage{wrapfig}
\usepackage{xeCJK}

\usetikzlibrary{mindmap,trees}

\newcommand{\rmnum}[1]{\romannumeral #1}
\newcommand{\HRule}{\rule{\linewidth}{0.5mm}}
\newcommand{\Rmnum}[1]{\expandafter\@slowromancap\romannumeral #1@}
\setCJKmonofont{SimSun}
\setCJKmainfont[BoldFont=SimHei]{SimSun}
\setsansfont{GoMono Nerd Font}
\setmainfont{GoMono Nerd Font}
\setmonofont[Mapping={}]{GoMono Nerd Font}

\graphicspath{{figures/}}
\allowdisplaybreaks

\setcounter{page}{1}

\begin{document}

\begin{center}
  \textbf{\Huge{第三章作业}}
\end{center}

\section*{3.59}
用x,y表示p分析:
\begin{align*}
  \because p &= x \times y \\
  &= (2^{ 64} \times x_h + x_l) \times (2 ^ {64} \times y_h + y_l) \\
  &= 2^{128} * x_h \times y_h + 2^{64} \times (x_h \times y_l + x_l \times y_h)
  + x_l \times y_l \\
  \therefore p_l &= x_l * y_l \\
  p_h &= x_h \times y_l + x_l \times y_h + (x_l \times y_l)_h
\end{align*}

反汇编代码分析:

\begin{lstlisting}[breaklines=true,captionpos=b,tabsize=2,numbers=left,columns=fullflexible,keepspaces=true]
# dest in %rdi, x in %rsi, y in %rdx
store_prod:
  movq %rdx, %rax     # %rax=y
  cqto                # expand sign bit of %rax(y) to %rdx
  # %rdx=y_h

  movq %rsi, %rcx     # %rcx=x
  sarq $63, %rcx      # %rcx=x >> 63
  # %rcx=x_h

  imulq %rax, %rcx    # %rcx=x_h*y_l
  imulq %rsi, %rdx    # %rdx=y_h*x_l
  addq %rdx, %rcx     # %rcx=x_h*y_l+x_l*y_h
  mulq %rsi
  # %rdx:%rax=x_l*y_l,%rax=(x_l*y_l)_l=p_l
  addq %rcx, %rdx
  # %rdx=x_h*y_l+x_l*y_h+(x_l*y_l)_h=p_h

  movq %rax, (%rdi)
  movq %rdx, 8(%rdi)
  # *dest=p_h:p_l
  ret
\end{lstlisting}

\section*{3.63}

\begin{lstlisting}[language=c++,breaklines=true,captionpos=b,tabsize=2,numbers=left,columns=fullflexible,keepspaces=true]
long switch_prob(long x, long n) {
  long result = x;
  switch (n) {
    case 60:
    case 62:
    result = x * 8;
    break;
    case 63:
    result = x / 8;
    break;
    case 64:
    result = x * 15;
    x = result;
    case 65:
    x *= x;
    default:
    result = x + 75;
  }
  return result;
}
\end{lstlisting}

\section*{3.67}

\subsubsection*{A}
\begin{lstlisting}[basicstyle=\small\ttfamily,extendedchars=false]
104  +------------------+
     |                  |
     |                  |
     |                  |
     |                  |
     |                  |
     |                  |
     |                  |
     |                  |
 64  +------------------+ <-- %rdi
     |                  |
     |                  |
     |                  |
     |                  |
     |                  |
     |                  |
 32  +------------------+
     |         z        |
 24  +------------------+
     |        &z        |
 16  +------------------+
     |         y        |
  8  +------------------+
     |         x        |
  0  +------------------+ <-- %rsp
\end{lstlisting}

\subsubsection*{B}
传入了一个\%rsp+64的指针。

\subsubsection*{C}
通过\%rsp指针加上偏移量。

\subsubsection*{D}
通过访问\%rsp+64指针加上偏移量。

\subsubsection*{E}
\begin{lstlisting}[basicstyle=\small\ttfamily,extendedchars=false]
104  +------------------+
     |                  |
     |                  |
 88  +------------------+
     |         z        |
 80  +------------------+
     |         x        |
 72  +------------------+
     |         y        |
 64  +------------------+ <-- %rdi
     |                  |   \
     |                  |    \
     |                  |      %rax
     |                  |
     |                  |
     |                  |
 32  +------------------+
     |         z        |
 24  +------------------+
     |        &z        |
 16  +------------------+
     |         y        |
  8  +------------------+
     |         x        |
  0  +------------------+ <-- %rsp
\end{lstlisting}

eval通过访问\%rsp加上偏移量访问r。

\subsubsection*{F}
如果把结构作为参数,那么实际传递的会是一个空的位置指针,函数把数据存储在这个位置
上,同时返回值也是这个指针。

\section*{3.71}

\begin{lstlisting}[language=c++]
#include <assert.h>
#include <stdio.h>

#define BUFFER_SIZE

int good_echo() {
  char buffer[BUFFER_SIZE];
  if (fgets(buffer, BUFFER_SIZE, stdin) == NULL) {
    return EOF;
  }
  puts(buffer);
}
\end{lstlisting}

\end{document}
