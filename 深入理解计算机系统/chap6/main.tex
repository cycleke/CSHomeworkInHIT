\documentclass[11pt]{article}

\usepackage[fleqn]{amsmath}
\usepackage{amssymb}
\usepackage[english]{babel}
\usepackage{bookmark}
\usepackage{booktabs}
\usepackage{capt-of}
\usepackage{colortbl}
\usepackage{dcolumn}
\usepackage{fancyhdr}
\usepackage[T1]{fontenc}
\usepackage{graphicx}
\usepackage{grffile}
\usepackage{hyperref}
\usepackage[utf8]{inputenc}
\usepackage{indentfirst}
\usepackage{listings}
\usepackage{longtable}
\usepackage{rotating}
\usepackage{textcomp}
\usepackage{tikz}
\usepackage[normalem]{ulem}
\usepackage{verbatim}
\usepackage{wrapfig}
\usepackage{xeCJK}

\usetikzlibrary{mindmap,trees}

\newcommand{\rmnum}[1]{\romannumeral #1}
\newcommand{\HRule}{\rule{\linewidth}{0.5mm}}
\newcommand{\Rmnum}[1]{\expandafter\@slowromancap\romannumeral #1@}
\setCJKmonofont{PingFang SC}
\setCJKmainfont{PingFang SC}
\setsansfont{GoMono Nerd Font}
\setmainfont{GoMono Nerd Font}
\setmonofont[Mapping={}]{GoMono Nerd Font}

\setlength{\parindent}{2em}

\lstset{
  breaklines=true,
  breakatwhitespace=false,
  captionpos=b,
  tabsize=2,
  numbers=left,
  numberstyle=\tiny\color{gray},
  columns=fullflexible,
  frame=shadowbox,
  keepspaces=true,
  commentstyle=\color[RGB]{0,128,0},
  keywordstyle=\color[RGB]{0,0,255},
  basicstyle=\footnotesize\ttfamily,
  rulesepcolor=\color{red!20!green!20!blue!20},
  showstringspaces = false,
}

\graphicspath{{figures/}}
\allowdisplaybreaks

\setcounter{page}{1}

\begin{document}

\begin{center}
  \textbf{\Huge{第六章作业}}
\end{center}

\section*{6.23}
\[T_{avg\ seek} = 4ms\]
\[T_{avg\ rotation} = \frac{1}{2} \times \frac{1}{RPM} \times \frac{60s}{1min}
  \approx 2ms \]
\[T_{avg\ transfer} = \frac{1}{RPM} \times \frac{1}{\text{(平均扇区数/磁道)}}
  \times \frac{60s}{1min} = 0.005ms\]
\[T_{access}=T_{avg\ seek}+T_{avg\ rotation}+T_{avg\ transfer} \approx 6.005ms\]

\section*{6.27}
\subsection*{A.}
t = 0x45 = 0b01000101, s = 0b001, b = ??。
所以可以命中的地址格式为 0 1000 1010 01??,内存地址范围为0x08A4 \textasciitilde
0x08A7。

t = 0x38同理有内存地址范围为0x0704 \textasciitilde 0x0707。
\subsection*{B.}
同理有内存地址范围为0x1238 \textasciitilde 0x123B。

\section*{6.31}
\subsection*{A.}
\begin{tabular}[htbp!]{|c|c|c|c|c|c|c|c|c|c|c|c|c|}
  \hline
  12 & 11 & 10 & 9 & 8 & 7 & 6 & 5 & 4 & 3 & 2 & 1 & 0 \\
  \hline
  0 &  0 &  1 & 1 & 1 & 0 & 0 & 0 & 1 & 1 & 0 & 1 & 0 \\
  \hline
\end{tabular}
\subsection*{B.}
\begin{tabular}[htbp!]{|c|c|}
  \hline
  参数 & 值 \\
  \hline
  高速缓存块偏移(CO) & 0x02 \\
  \hline
  高速缓存组索引(CI) & 0x06 \\
  \hline
  高速缓存标记(CT) & 0x38 \\
  \hline
  高速是否命中?(是/否) & 是 \\
  \hline
  返回的高速缓存字节 & 0xEB \\
  \hline
\end{tabular}

\section*{6.35}
\begin{tabular}[htbp!]{|c|c|c|c|c|}
  \multicolumn{5}{c}{dst数组} \\
  \hline
  & 列0 & 列1 & 列2 & 列3 \\
  \hline
  行0 & m & h & h & h \\
  \hline
  行1 & m & h & h & h \\
  \hline
  行2 & m & h & h & h \\
  \hline
  行3 & m & h & h & h \\
  \hline
\end{tabular}
\begin{tabular}[htbp!]{|c|c|c|c|c|}
  \multicolumn{5}{c}{src数组} \\
  \hline
  & 列0 & 列1 & 列2 & 列3 \\
  \hline
  行0 & m & h & h & h \\
  \hline
  行1 & m & h & h & h \\
  \hline
  行2 & m & h & h & h \\
  \hline
  行3 & m & h & h & h \\
  \hline
\end{tabular}

\section*{6.39}
\subsection*{A.}
总写数=16*16*4=1024
\subsection*{B.}
sizeof(point\_color)==16,B=32,所以第一个c无法命中,剩下的m、y、k均命中。步长为
16*16=256,所以square[j+1][i]不会命中,也不会覆盖上一次写入的高速缓存。
2048/256=8,所以square[j+8][i]会覆盖square[j][i]写入的高速缓存,所以
square[j][i+1]有不会命中。

不命中写总数$=\frac{1}{4} \times 16 * 16 * 4 = 256$
\subsection*{C.}
不命中率$=\frac{1}{4}$。

\section*{6.43}
由于每行四个字节,而sizeof(int)==4,所以不命中率为100\%。
\end{document}
