\documentclass[11pt]{article}

\usepackage[fleqn]{amsmath}
\usepackage{amssymb}
\usepackage[english]{babel}
\usepackage{bookmark}
\usepackage{booktabs}
\usepackage{capt-of}
\usepackage{colortbl}
\usepackage{dcolumn}
\usepackage{fancyhdr}
\usepackage[T1]{fontenc}
\usepackage{graphicx}
\usepackage{grffile}
\usepackage{hyperref}
\usepackage[utf8]{inputenc}
\usepackage{listings}
\usepackage{longtable}
\usepackage{rotating}
\usepackage{textcomp}
\usepackage{tikz}
\usepackage[normalem]{ulem}
\usepackage{verbatim}
\usepackage{wrapfig}
\usepackage{xeCJK}

\usetikzlibrary{mindmap,trees}

\newcommand{\rmnum}[1]{\romannumeral #1}
\newcommand{\HRule}{\rule{\linewidth}{0.5mm}}
\newcommand{\Rmnum}[1]{\expandafter\@slowromancap\romannumeral #1@}
\setCJKmonofont{PingFang SC}
\setCJKmainfont{PingFang SC}
\setsansfont{GoMono Nerd Font}
\setmainfont{GoMono Nerd Font}
\setmonofont[Mapping={}]{GoMono Nerd Font}
\lstset{
  breaklines=true,
  breakatwhitespace=false,
  captionpos=b,
  tabsize=2,
  numbers=left,
  numberstyle=\tiny\color{gray},
  columns=fullflexible,
  frame=shadowbox,
  keepspaces=true,
  commentstyle=\color[RGB]{0,128,0},
  keywordstyle=\color[RGB]{0,0,255},
  basicstyle=\footnotesize\ttfamily,
  rulesepcolor=\color{red!20!green!20!blue!20},
  showstringspaces = false,
}

\graphicspath{{figures/}}
\allowdisplaybreaks

\setcounter{page}{1}

\begin{document}

\begin{center}
  \textbf{\Huge{第五章作业}}
\end{center}

\section*{5.15}

循环展开如下:
\begin{lstlisting}[language=c]
void inner4(vec_ptr u, vec_ptr v, data_t *dest) {
  long i;
  long length = vec_length(u);
  data_t *udata = get_vec_start(u);
  data_t *vdata = get_vec_start(v);
  data_t sum = (data_t) 0;
  data_t sum1 = (data_t) 0;
  data_t sum2 = (data_t) 0;
  data_t sum3 = (data_t) 0;
  data_t sum4 = (data_t) 0;
  data_t sum5 = (data_t) 0;

  for (i = 0; i < length-6; i+=6) {
    sum = sum + udata[i] * vdata[i];
    sum1 = sum1 + udata[i+1] * vdata[i+1];
    sum2 = sum2 + udata[i+2] * vdata[i+2];
    sum3 = sum3 + udata[i+3] * vdata[i+3];
    sum4 = sum4 + udata[i+4] * vdata[i+4];
    sum5 = sum5 + udata[i+5] * vdata[i+5];
  }
  for(; i < length; ++i)
    sum = sum + udata[i] * vdata[i];
  *dest = sum + sum1 + sum2 + sum3 + sum4 + sum5;
}
\end{lstlisting}
处理器的浮点数乘法的容量只有2(<6),限制了CPE。

\section*{5.19}
\begin{lstlisting}[language=c]
void psum_4_1a(float a[], float p[], long n) {
  long i;
  float val, last_val;
  float tmp, tmp1, tmp2, tmp3;
  last_val = p[0] = a[0];

  for (i = 1; i < n - 4; i++) {
    tmp = last_val + a[i];
    tmp1 = tmp + a[i+1];
    tmp2 = tmp1 + a[i+2];
    tmp3 = tmp2 + a[i+3];

    p[i] = tmp;
    p[i+1] = tmp1;
    p[i+2] = tmp2;
    p[i+3] = tmp3;

    /* key point */
    last_val = last_val + (a[i] + a[i+1] + a[i+2] + a[i+3]);
  }

  for (; i < n; i++) {
    last_val += a[i];
    p[i] = last_val;
  }
}
\end{lstlisting}

\end{document}
