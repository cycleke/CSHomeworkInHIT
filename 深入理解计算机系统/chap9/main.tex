\documentclass[11pt]{article}

\usepackage[fleqn]{amsmath}
\usepackage{amssymb}
\usepackage[english]{babel}
\usepackage{bookmark}
\usepackage{booktabs}
\usepackage{capt-of}
\usepackage{colortbl}
\usepackage{dcolumn}
\usepackage{fancyhdr}
\usepackage[T1]{fontenc}
\usepackage{graphicx}
\usepackage{grffile}
\usepackage{hyperref}
\usepackage[utf8]{inputenc}
\usepackage{indentfirst}
\usepackage{listings}
\usepackage{longtable}
\usepackage{rotating}
\usepackage{textcomp}
\usepackage{tikz}
\usepackage[normalem]{ulem}
\usepackage{verbatim}
\usepackage{wrapfig}
\usepackage{xeCJK}

\usetikzlibrary{mindmap,trees}

\newcommand{\rmnum}[1]{\romannumeral #1}
\newcommand{\HRule}{\rule{\linewidth}{0.5mm}}
\newcommand{\Rmnum}[1]{\expandafter\@slowromancap\romannumeral #1@}
\setCJKmonofont{PingFang SC}
\setCJKmainfont{PingFang SC}
\setsansfont{GoMono Nerd Font}
\setmainfont{GoMono Nerd Font}
\setmonofont[Mapping={}]{GoMono Nerd Font}

\setlength{\parindent}{2em}

\lstset{
  breaklines=true,
  breakatwhitespace=false,
  captionpos=b,
  tabsize=2,
  numbers=left,
  numberstyle=\tiny\color{gray},
  columns=fullflexible,
  frame=shadowbox,
  keepspaces=true,
  commentstyle=\color[RGB]{0,128,0},
  keywordstyle=\color[RGB]{0,0,255},
  basicstyle=\footnotesize\ttfamily,
  rulesepcolor=\color{red!20!green!20!blue!20},
  showstringspaces = false,
}

\graphicspath{{figures/}}
\allowdisplaybreaks

\setcounter{page}{1}

\begin{document}

\begin{center}
  \textbf{\Huge{第九章作业}}
\end{center}

\section*{9.11}
\subsection*{A.虚拟地址格式}
\begin{tabular}[htbp!]{|c|c|c|c|c|c|c|c|c|c|c|c|c|c|}
  \toprule
  13 & 12 & 11 & 10 & 9 & 8 & 7 & 6 & 5 & 4 & 3 & 2 & 1 & 0 \\
  \midrule
  0 &  0 &  0 &  0 & 1 & 0 & 0 & 1 & 1 & 1 & 1 & 1 & 0 & 0 \\
  \bottomrule
\end{tabular}

\subsection*{B.地址翻译}
\begin{tabular}[htbp!]{|c|c|}
  \toprule
  参数 & 值 \\\midrule
  VPN & 0x9 \\\midrule
  TLB索引 & 0x1 \\\midrule
  TLB标记 & 0x2 \\\midrule
  TLB命中 & 否 \\\midrule
  缺页 & 否 \\\midrule
  PPN & 0x17 \\\bottomrule
\end{tabular}

\subsection*{C.物理地址格式}
\begin{tabular}[htbp!]{|c|c|c|c|c|c|c|c|c|c|c|c|}
  \toprule
  11 & 10 & 9 & 8 & 7 & 6 & 5 & 4 & 3 & 2 & 1 & 0 \\
  \midrule
  0 &  1 & 0 & 1 & 1 & 1 & 1 & 1 & 1 & 1 & 0 & 0 \\
  \bottomrule
\end{tabular}

\subsection*{D.物理地址引用}
\begin{tabular}[htbp!]{|c|c|}
  \toprule
  参数 & 值 \\\midrule
  字节偏移 & 0x0 \\\midrule
  缓存索引 & 0xf \\\midrule
  缓存标记 & 0x17 \\\midrule
  缓存命中 & 否 \\\midrule
  返回的缓存字节 & - \\\bottomrule
\end{tabular}

\section*{9.15}
\begin{tabular}[htbp!]{|c|c|c|}
  \toprule
  请求 & 块大小(十进制) & 块头部(十六进制) \\\midrule
  malloc(3) & 8 & 0x9 \\\midrule
  malloc(11) & 16 & 0x11 \\\midrule
  malloc(20) & 24 & 0x19 \\\midrule
  malloc(21) & 24 & 0x19 \\\bottomrule
\end{tabular}

\section*{9.19}
1. a; 2. d; 3. b;

\end{document}
