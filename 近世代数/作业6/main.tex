% !Tex Program = xelatex
% -*-coding: utf-8 -*-
\documentclass[12pt,onecolumn]{article}

% 中文
\usepackage[BoldFont,SlantFont]{xeCJK}
\xeCJKsetemboldenfactor{1}%只对随后定义的CJK字体有效
\setCJKfamilyfont{hei}{SimHei}
\xeCJKsetemboldenfactor{4}
\setCJKfamilyfont{song}{SimSun}
\xeCJKsetemboldenfactor{4}
\setCJKfamilyfont{fs}{FangSong}
\setCJKfamilyfont{kai}{KaiTi}
\setCJKfamilyfont{li}{LiSu}
\setCJKfamilyfont{xw}{STXinwei}
\setCJKmainfont{SimSun}

\newcommand{\hei}{\CJKfamily{hei}}      % 黑体
\newcommand{\song}{\CJKfamily{song}}    % 宋体   (Windows 自带simsun.ttf)
\newcommand{\fs}{\CJKfamily{fs}}        % 仿宋体 (Windows 自带simfs.ttf)
\newcommand{\kai}{\CJKfamily{kai}}      % 楷体   (Windows 自带simkai.ttf)
\newcommand{\li}{\CJKfamily{li}}        % 隶书   (Windows自带simli.ttf)
\newcommand{\xw}{\CJKfamily{xw}}        % 隶书   (Windows自带simli.ttf)

% \AmSTeX\ 宏包,用来排出更加漂亮的公式。
\usepackage{amsmath}
% 定理类环境宏包,其中 \pkg{amsmath} 选项用来兼容 \AmSTeX\ 的宏包
\usepackage[amsmath,thmmarks,hyperref]{ntheorem}
\usepackage{amssymb}
% 添加字体
\usepackage[defaultsups]{newtxtext}
\usepackage{newtxmath}
\usepackage{courier}
% 图形支持宏包
\usepackage{graphicx}
% 插入pdf
\usepackage{pdfpages}
\includepdfset{fitpaper=true}
% 更好的列表环境。
\usepackage{enumitem}       %使用enumitem宏包,改变列表项的格式
\usepackage{enumerate}
\usepackage{environ}
% 禁止 \LaTeX 自动调整多余的页面底部空白,并保持脚注仍然在底部。
% 脚注按页编号。
\usepackage[bottom,perpage,hang]{footmisc}
\raggedbottom
% 脚注格式。
\usepackage{pifont}
% 表格控制
\usepackage{longtable}
\usepackage{booktabs}
% 参考文献引用宏包
\usepackage[sort&compress]{natbib}
% 生成有书签的 pdf 及其开关,请结合 gbk2uni 避免书签乱码。
\usepackage{hyperref}
\hypersetup{
  CJKbookmarks=true,
  linktoc=all,
  bookmarksnumbered=true,
  bookmarksopen=true,
  bookmarksopenlevel=1,
  breaklinks=true,
  colorlinks=false,
  plainpages=false,
  pdfborder=0 0 0}
% 设置 url 样式,与上下文一致
\urlstyle{same}
% 版芯设置
\usepackage{geometry}
\geometry{
  centering,
  text={150true mm,236true mm},
  left=30true mm,
  head=5true mm,
  headsep=2true mm,
  footskip=0true mm,
  foot=5.2true mm
}
% 利用 \pkg{fancyhdr} 设置页眉页脚。
\usepackage{fancyhdr}
% 其他包,表格、数学符号包
\usepackage{tabularx}
\usepackage{varwidth}
% 此处changepage环境用来控制索引页面的左右边距,规范中给出的示例的边距要大于正文。
\usepackage{changepage}
% 多栏结构在文中用begin{multicols}{2}end{multicols}
\usepackage{multicol,multienum}
% 允许上一个section的浮动图形出现在下一个section的开始部分,还提供\FloatBarrier命
% 令,使所有未处理的浮动图形立即被处理
\usepackage[below]{placeins}
% 支持子图 %centerlast 设置最后一行是否居中
\usepackage{subfigure}
% 支持双语标题
\usepackage[subfigure]{ccaption}
% 根据我工规定,正文小四号 (12bp) 字,行距为固定值3--4mm。
\renewcommand\normalsize{%
  % \@setfontsize\normalsize{12bp}{\ifhit@glue 20.50398bp \@plus 2.83465bp \@minus 0bp\else 20.50398bp\fi}%
  \abovedisplayskip=8pt
  \abovedisplayshortskip=8pt
  \belowdisplayskip=\abovedisplayskip
  \belowdisplayshortskip=\abovedisplayshortskip}
% 根据习惯定义字号。用法:\cs{hit@def@fontsize}\marg{字号名称}\marg{磅数}避免了
% 字号选择和行距的紧耦合。所有字号定义时为单倍行距,并提供选项指定行距倍数。
\def\hit@def@fontsize#1#2{%
  \expandafter\newcommand\csname #1\endcsname[1][1.3]{%
    \fontsize{#2}{##1\dimexpr #2}\selectfont}}
\hit@def@fontsize{dachu}{58bp}
\hit@def@fontsize{chuhao}{42bp}
\hit@def@fontsize{xiaochu}{36bp}
\hit@def@fontsize{yihao}{26bp}
\hit@def@fontsize{xiaoyi}{24bp}
\hit@def@fontsize{erhao}{22bp}
\hit@def@fontsize{xiaoer}{18bp}
\hit@def@fontsize{sanhao}{16bp}
\hit@def@fontsize{xiaosan}{15bp}
\hit@def@fontsize{sihao}{14bp}
\hit@def@fontsize{banxiaosi}{13bp}
\hit@def@fontsize{xiaosi}{12bp}
\hit@def@fontsize{dawu}{11bp}
\hit@def@fontsize{wuhao}{10.5bp}
\hit@def@fontsize{xiaowu}{9bp}
\hit@def@fontsize{liuhao}{7.5bp}
\hit@def@fontsize{xiaoliu}{6.5bp}
\hit@def@fontsize{qihao}{5.5bp}
\hit@def@fontsize{bahao}{5bp}
% 利用 \pkg{enumitem} 命令调整默认列表环境间的距离,以符合中文习惯。
\setlist{nosep}
% 允许太长的公式断行、分页等。
\allowdisplaybreaks[4]
\predisplaypenalty=0  %公式之前可以换页,公式出现在页面顶部
\postdisplaypenalty=0
% 公式编号设置
\renewcommand{\theequation}{\arabic{section}.\arabic{equation}}
% 定理标题使用黑体,正文使用宋体,冒号隔开。
\theorembodyfont{\normalfont}
\theoremheaderfont{\normalfont\hei}
\theoremsymbol{\ensuremath{\square}}
\newtheorem*{proof}{证明}
\theoremstyle{plain}
\theoremsymbol{}
\theoremseparator{}
\newtheorem{assumption}{假设}[section]
\newtheorem{definition}{定义}[section]
\newtheorem{proposition}{命题}[section]
\newtheorem{lemma}{引理}[section]
\newtheorem{theorem}{定理}[section]
\newtheorem{axiom}{公理}[section]
\newtheorem{corollary}{推论}[section]
\newtheorem{exercise}{练习}[section]
\newtheorem{example}{例}[section]
\newtheorem{remark}{注释}[section]
\newtheorem{problem}{问题}[section]
\newtheorem{conjecture}{猜想}[section]
% 各种单位
\usepackage{siunitx}
\sisetup{group-minimum-digits=4, group-separator= \hspace{0.25em}}
\sisetup{detect-weight,detect-mode,detect-family}
% 处理数学公式中的黑斜体的宏包
\usepackage{bm}
% 不同于 \mathcal \mathfrak 之类的英文花体字体
\usepackage{mathrsfs}
% 支持彩色
\usepackage{xcolor}
\definecolor{colorzero}{rgb}{0, 0, 0}
\definecolor{colorone}{rgb}{1, 0, 0}
\definecolor{colortwo}{rgb}{0, 0, 1}
\definecolor{colorthree}{rgb}{0, 1, 0}
% 图形和表格的控制旋转
\usepackage{rotating}
% 算法的宏包,注意宏包兼容性,先后顺序为float、hyperref、algorithm(2e),否则无法
% 生成算法列表。
\usepackage[algoruled,linesnumbered]{algorithm2e}
% 排版源码所使用的环境。
\usepackage{listings}
\lstset{
  breaklines  = true,
  captionpos  = b,
  tabsize     = 2,
  numbers     = left,
  columns     = flexible,
  keepspaces  = true,
  % commentstyle = \color[RGB]{0,128,0},
  % keywordstyle = \color[RGB]{0,0,255},
  basicstyle   = \small\ttfamily,
  rulesepcolor = \color{red!20!green!20!blue!20},
  showstringspaces = false,
}

% 作图
\usepackage{tikz}

% 首行缩进
\usepackage{indentfirst}
\setlength{\parindent}{2em}

% 最后定义一些常见的数学公式样式。
\newcommand{\theVector}[1]{\bm{#1}}
\newcommand{\theMatrix}[1]{\mathbb{#1}}
\newcommand{\theSet}[1]{\mathcal{#1}}
\newcommand{\theDirected}[1]{{\overrightarrow{#1}}}
\newcommand{\theUndirected}[1]{{\overline{#1}}}
\newcommand{\theNetwork}[1]{\mathscr{#1}}
\newcommand{\theNode}[1]{{\text{#1}}}
\newcommand{\theDirectedEdge}[2]{{\overrightarrow{{#1}{#2}}}}
\newcommand{\theUndirectedEdge}[2]{{\overline{{#1}{#2}}}}

\graphicspath{{figures/}}

\title{近世代数作业6}
\author{cycleke}

% \renewcommand{\thesection}{}
% \renewcommand{\thesubsection}{}

\begin{document}
\maketitle
\tableofcontents
\clearpage

\section{课后习题}

\subsection{第一题}
\begin{proof*}
  由于$A$,$B$为$G$的有限子群,所以$A \cap B$为$A$和$B$的子群。
  设$H = A \cap B$,$S_l = \{a_1H, a_2H, \ldots, a_k H\}$为$H$的所有不同的左陪集之集,其中$a_i \in A, i = 1, 2, \ldots, k$。
  所以$S_l$为$A$的一个划分,所以$|A| = k|H|$。

  因为$\forall a_i, a_j(i \neq j)$,$a_iH \neq a_j H$,所以$a_i a_j^{-1} \notin H$,然而$a_i a_j^{-1} \in A$,所以$a_i a_j^{-1} \notin B$。

  又因为对于$AB$,有
  \begin{align*}
    AB & = (\bigcup_{i = 1}^{k} a_i H) B         \\
       & = \bigcup_{i = 1}^{k}(a_i H B)          \\
       & = \bigcup_{i = 1}^{k}(a_i (A \cap B) B) \\
       & = \bigcup_{i = 1}^{k}(a_i B)            \\
  \end{align*}

  所以$\forall a_i, a_j(i \neq j)$,$a_i B \neq a_j B$(否则有$a_i a_j^{-1} \in B$,矛盾)。
  所以$a_1 B, a_2 B, \ldots, a_k B$为$AB$的一个划分,进而有$|AB| = k|B|$。

  综上有,$|AB| = k|B| = \frac{|A||B|}{|H|} = \frac{|A||B|}{|A \cap B|}$。
\end{proof*}

\subsection{第二题}
当$n = 1$时,命题并不成立,下面给出$n \geq 2$时的证明。

\begin{proof*}
  设$P = x^{-1}Hx$
  首先证明$P \leq G$。
  由于$e \in H$所以$e = x^{-1}ex \in x^{-1}Hx$。我们只需证明封闭性和逆元存在。

  对于封闭性,有
  $\forall p_1, p_2 \in P, \exists h_1, h_2 \in H, p_1 = x^{-1}h_1x, p_2 = x^{-1}h_2x , h_1h_2 \in H,\text{所以}p_1p_2 = (x^{-1}h_1x)(x^{-1}h_2x) = x^{-1}(h_1h_2)x \in P$。

  对于逆元存在,有$\forall p \in P, \exists h \in H, p = x^{-1}hx, \text{因为}x^{-1}h^{-1}x \in P,\text{而} p(x^{-1}h^{-1}x) = e$,所以每个元素均有逆元存在。

  所以$ P \leq G$,且$P_0$与$H$内自同构,$|P| = |H| = n$。

  假设 $\exists x_0 \in G, s.t. x_0^{-1}Hx_0 \cap H = \{e\}$。设$P_0 = x_0^{-1}Hx_0$,
  则$P_0$与$H$内自同构,$|P_0| = n$。

  $\forall a \in H, b \in P_0$,且a,b不同时为e,则有$ab \notin H$且$ab \notin P_0$(否则$a \in P_0\text{或}b \in H \Rightarrow P \cap H \neq \{e\}$)。
  所以$|HP_0| = |H||P_0| = n^2 = |G|$,由于$ab \in G$,所以$HP_0 = G$。由例12.7.1知$HP_0 \neq \{e\}$,矛盾。

  综上所述,$\forall x \in G, x^{-1}Hx \cap H \neq \{e\}$。
\end{proof*}

\subsection{第三题}
\begin{proof*}
  由前面的习题知三阶子群存在。

  若三阶子群不唯一,不妨设$A,B$为六阶群$G$两个不同的三阶子群。则$|AB| = \frac{9}{|A \cap B|}$。
  由于$A$,$B$为两个不同的三阶子群,所以$|A \cap B| = 1 \Rightarrow |AB| = 9 > 6$,产生矛盾。

  所以六阶群中有唯一一个三阶子群。
\end{proof*}

\subsection{第四题}
\begin{proof*}
  设有群$G$,$H$为$G$的一个子群且$[G:H] = 2$。则在$G$中,$H$有且只有两个左陪集$H$和$G-H$。

  $\forall a \in H$,$aH = H = Ha$。

  $\forall a \in G - H$,$aH = G - H = Ha$。

  所以$H$为$G$的一个正规子群。
\end{proof*}

\subsection{第五题}
\begin{proof}
  设$A$,$B$为$G$的两个正规子群,$C = A \cap B$。显然$C$为$G$的一个子群。
  $\forall g \in G, \forall c \in C, \text{有} gcg^{-1} \in A, gcg^{-1} \in B, \text{所以} gcg^{-1} \in C, \text{所以} gCg^{-1} \in C$。
  所以$C$为$G$的一个正规子群。
\end{proof}

\subsection{第六题}
\begin{proof*}
  由于$N$是群$G$的子群,$H$是$G$的正规子群,所以$\forall h \in H, Nh = hN$,即$\forall x \in NH, \exists y \in HN, x = y$。
  所以$NH \subseteq HN$,同理有$HN \subseteq NH$。所以$NH \leq G$。
\end{proof*}

\subsection{第七题}
\begin{proof*}
  由于$G$为有限交换群,拉格朗日定理的逆命题成立,所以$G$有一个二阶子群,而$G$对其的划分为一个$n$阶商群。
\end{proof*}

\subsection{第八题}
\begin{proof*}
  $\Rightarrow$. 由于$H \triangleleft G$,所以$H$的左陪集集族构成商群,所以其任两个左陪集的乘积还是一个左陪集。

  $\Leftarrow$. $\forall a, b \in G, \exists c \in G, aHbH = cH$。因为$e \in H$,所以$ab = aebe \in cH$。
  进而有$abH = cH$,$aHbH = abH$。所以$\forall a \in G, aHa^{-1}H = H$,所以$aHa^{-1} = aHa^{-1}e \in H$。
  所以$H \triangleleft G$。
\end{proof*}

\subsection{第九题}
\begin{proof*}
  不妨设$H = {e, a}, a^2 = e\forall x \in G, xHx^{-1} = H$。
  因为$xex^{-1} = e$,所以$xax^{-1} = a$。所以$e, a \in C, H \subseteq C$。
\end{proof*}

\subsection{第十题}
\begin{proof*}
  设G为群,$\forall H \leq G, G^{(1)} \subseteq H$,显然$G^{(1)} \leq H$。
  $\forall x \in G, h \in H, xhx^{-1}h^{-1} \in G^{(1)} \leq H$。进而有$xhx^{-1} = (xhx^{-1}h^{-1})h \in H$,
  所以$xHx^{-1} \leq H$,所以$H \triangleleft G$。
\end{proof*}

\subsection{第十一题}
\begin{proof*}
  先证$AB \subseteq G$,显然。

  再证$G \subseteq AB$。$\forall g \in G, |A^{-1}g| = |A|$,所以$|A^{-1}g| + |B| = |A| + |B| > |G|$,
  所以$A^{-1}g \cap B \neq \emptyset$.于是$\exists a \in A, b \in B, s.t. a^{-1}g = b, g = ab \in AB$,
  所以$G \subseteq AB$。

  综上所述,$AB = G$。
\end{proof*}

\end{document}

%%% Local Variables:
%%% mode: latex
%%% TeX-master: t
%%% End:
