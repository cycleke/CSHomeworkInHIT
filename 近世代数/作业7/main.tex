% !Tex Program = xelatex
% -*-coding: utf-8 -*-
\documentclass[12pt,onecolumn]{article}

% 中文
\usepackage[BoldFont,SlantFont]{xeCJK}
\xeCJKsetemboldenfactor{1}%只对随后定义的CJK字体有效
\setCJKfamilyfont{hei}{SimHei}
\xeCJKsetemboldenfactor{4}
\setCJKfamilyfont{song}{SimSun}
\xeCJKsetemboldenfactor{4}
\setCJKfamilyfont{fs}{FangSong}
\setCJKfamilyfont{kai}{KaiTi}
\setCJKfamilyfont{li}{LiSu}
\setCJKfamilyfont{xw}{STXinwei}
\setCJKmainfont{SimSun}

\newcommand{\hei}{\CJKfamily{hei}}      % 黑体
\newcommand{\song}{\CJKfamily{song}}    % 宋体   (Windows 自带simsun.ttf)
\newcommand{\fs}{\CJKfamily{fs}}        % 仿宋体 (Windows 自带simfs.ttf)
\newcommand{\kai}{\CJKfamily{kai}}      % 楷体   (Windows 自带simkai.ttf)
\newcommand{\li}{\CJKfamily{li}}        % 隶书   (Windows自带simli.ttf)
\newcommand{\xw}{\CJKfamily{xw}}        % 隶书   (Windows自带simli.ttf)

% \AmSTeX\ 宏包,用来排出更加漂亮的公式。
\usepackage{amsmath}
% 定理类环境宏包,其中 \pkg{amsmath} 选项用来兼容 \AmSTeX\ 的宏包
\usepackage[amsmath,thmmarks,hyperref]{ntheorem}
\usepackage{amssymb}
% 添加字体
\usepackage[defaultsups]{newtxtext}
\usepackage{newtxmath}
\usepackage{courier}
% 图形支持宏包
\usepackage{graphicx}
% 插入pdf
\usepackage{pdfpages}
\includepdfset{fitpaper=true}
% 更好的列表环境。
\usepackage{enumitem}       %使用enumitem宏包,改变列表项的格式
\usepackage{enumerate}
\usepackage{environ}
% 禁止 \LaTeX 自动调整多余的页面底部空白,并保持脚注仍然在底部。
% 脚注按页编号。
\usepackage[bottom,perpage,hang]{footmisc}
\raggedbottom
% 脚注格式。
\usepackage{pifont}
% 表格控制
\usepackage{longtable}
\usepackage{booktabs}
% 参考文献引用宏包
\usepackage[sort&compress]{natbib}
% 生成有书签的 pdf 及其开关,请结合 gbk2uni 避免书签乱码。
\usepackage{hyperref}
\hypersetup{
  CJKbookmarks=true,
  linktoc=all,
  bookmarksnumbered=true,
  bookmarksopen=true,
  bookmarksopenlevel=1,
  breaklinks=true,
  colorlinks=false,
  plainpages=false,
  pdfborder=0 0 0}
% 设置 url 样式,与上下文一致
\urlstyle{same}
% 版芯设置
\usepackage{geometry}
\geometry{
  centering,
  text={150true mm,236true mm},
  left=30true mm,
  head=5true mm,
  headsep=2true mm,
  footskip=0true mm,
  foot=5.2true mm
}
% 利用 \pkg{fancyhdr} 设置页眉页脚。
\usepackage{fancyhdr}
% 其他包,表格、数学符号包
\usepackage{tabularx}
\usepackage{varwidth}
% 此处changepage环境用来控制索引页面的左右边距,规范中给出的示例的边距要大于正文。
\usepackage{changepage}
% 多栏结构在文中用begin{multicols}{2}end{multicols}
\usepackage{multicol,multienum}
% 允许上一个section的浮动图形出现在下一个section的开始部分,还提供\FloatBarrier命
% 令,使所有未处理的浮动图形立即被处理
\usepackage[below]{placeins}
% 支持子图 %centerlast 设置最后一行是否居中
\usepackage{subfigure}
% 支持双语标题
\usepackage[subfigure]{ccaption}
% 根据我工规定,正文小四号 (12bp) 字,行距为固定值3--4mm。
\renewcommand\normalsize{%
  % \@setfontsize\normalsize{12bp}{\ifhit@glue 20.50398bp \@plus 2.83465bp \@minus 0bp\else 20.50398bp\fi}%
  \abovedisplayskip=8pt
  \abovedisplayshortskip=8pt
  \belowdisplayskip=\abovedisplayskip
  \belowdisplayshortskip=\abovedisplayshortskip}
% 根据习惯定义字号。用法:\cs{hit@def@fontsize}\marg{字号名称}\marg{磅数}避免了
% 字号选择和行距的紧耦合。所有字号定义时为单倍行距,并提供选项指定行距倍数。
\def\hit@def@fontsize#1#2{%
  \expandafter\newcommand\csname #1\endcsname[1][1.3]{%
    \fontsize{#2}{##1\dimexpr #2}\selectfont}}
\hit@def@fontsize{dachu}{58bp}
\hit@def@fontsize{chuhao}{42bp}
\hit@def@fontsize{xiaochu}{36bp}
\hit@def@fontsize{yihao}{26bp}
\hit@def@fontsize{xiaoyi}{24bp}
\hit@def@fontsize{erhao}{22bp}
\hit@def@fontsize{xiaoer}{18bp}
\hit@def@fontsize{sanhao}{16bp}
\hit@def@fontsize{xiaosan}{15bp}
\hit@def@fontsize{sihao}{14bp}
\hit@def@fontsize{banxiaosi}{13bp}
\hit@def@fontsize{xiaosi}{12bp}
\hit@def@fontsize{dawu}{11bp}
\hit@def@fontsize{wuhao}{10.5bp}
\hit@def@fontsize{xiaowu}{9bp}
\hit@def@fontsize{liuhao}{7.5bp}
\hit@def@fontsize{xiaoliu}{6.5bp}
\hit@def@fontsize{qihao}{5.5bp}
\hit@def@fontsize{bahao}{5bp}
% 利用 \pkg{enumitem} 命令调整默认列表环境间的距离,以符合中文习惯。
\setlist{nosep}
% 允许太长的公式断行、分页等。
\allowdisplaybreaks[4]
\predisplaypenalty=0  %公式之前可以换页,公式出现在页面顶部
\postdisplaypenalty=0
% 公式编号设置
\renewcommand{\theequation}{\arabic{section}.\arabic{equation}}
% 定理标题使用黑体,正文使用宋体,冒号隔开。
\theorembodyfont{\normalfont}
\theoremheaderfont{\normalfont\hei}
\theoremsymbol{\ensuremath{\square}}
\newtheorem*{proof}{证明}
\theoremstyle{plain}
\theoremsymbol{}
\theoremseparator{}
\newtheorem{assumption}{假设}[section]
\newtheorem{definition}{定义}[section]
\newtheorem{proposition}{命题}[section]
\newtheorem{lemma}{引理}[section]
\newtheorem{theorem}{定理}[section]
\newtheorem{axiom}{公理}[section]
\newtheorem{corollary}{推论}[section]
\newtheorem{exercise}{练习}[section]
\newtheorem{example}{例}[section]
\newtheorem{remark}{注释}[section]
\newtheorem{problem}{问题}[section]
\newtheorem{conjecture}{猜想}[section]
% 各种单位
\usepackage{siunitx}
\sisetup{group-minimum-digits=4, group-separator= \hspace{0.25em}}
\sisetup{detect-weight,detect-mode,detect-family}
% 处理数学公式中的黑斜体的宏包
\usepackage{bm}
% 不同于 \mathcal \mathfrak 之类的英文花体字体
\usepackage{mathrsfs}
% 支持彩色
\usepackage{xcolor}
\definecolor{colorzero}{rgb}{0, 0, 0}
\definecolor{colorone}{rgb}{1, 0, 0}
\definecolor{colortwo}{rgb}{0, 0, 1}
\definecolor{colorthree}{rgb}{0, 1, 0}
% 图形和表格的控制旋转
\usepackage{rotating}
% 算法的宏包,注意宏包兼容性,先后顺序为float、hyperref、algorithm(2e),否则无法
% 生成算法列表。
\usepackage[algoruled,linesnumbered]{algorithm2e}
% 排版源码所使用的环境。
\usepackage{listings}
\lstset{
  breaklines  = true,
  captionpos  = b,
  tabsize     = 2,
  numbers     = left,
  columns     = flexible,
  keepspaces  = true,
  % commentstyle = \color[RGB]{0,128,0},
  % keywordstyle = \color[RGB]{0,0,255},
  basicstyle   = \small\ttfamily,
  rulesepcolor = \color{red!20!green!20!blue!20},
  showstringspaces = false,
}

% 作图
\usepackage{tikz}

% 首行缩进
\usepackage{indentfirst}
\setlength{\parindent}{2em}

\usepackage{float}
\usepackage{diagbox}

% 最后定义一些常见的数学公式样式。
\newcommand{\theVector}[1]{\bm{#1}}
\newcommand{\theMatrix}[1]{\mathbb{#1}}
\newcommand{\theSet}[1]{\mathcal{#1}}
\newcommand{\theDirected}[1]{{\overrightarrow{#1}}}
\newcommand{\theUndirected}[1]{{\overline{#1}}}
\newcommand{\theNetwork}[1]{\mathscr{#1}}
\newcommand{\theNode}[1]{{\text{#1}}}
\newcommand{\theDirectedEdge}[2]{{\overrightarrow{{#1}{#2}}}}
\newcommand{\theUndirectedEdge}[2]{{\overline{{#1}{#2}}}}

\graphicspath{{figures/}}

\pagestyle{fancy}
\fancyhead[L]{\song\xiaowu[0]{哈尔滨工业大学}}
\fancyhead[R]{\song\xiaowu[0]{近世代数}}
\fancyfoot[C]{\xiaowu-~\thepage~-}

% \renewcommand{\thesection}{}
% \renewcommand{\thesubsection}{}
\renewcommand{\today}{\number\year{年}\number\month{月}\number\day{日}}
\renewcommand{\figurename}{图}
\renewcommand{\tablename}{表}

\title{近世代数作业七}
\author{cycleke}
\date{\today}


\begin{document}
\maketitle
\tableofcontents
\clearpage

\section{课后习题}

\subsection{第一题}
\begin{proof*}
  不妨设$G = \{e_1, a, a^2, \ldots, a^{m - 1}\}, \bar{G} = \{e_2, b, b^2, \ldots, a^{n - 1}\}$。

  先证$\Rightarrow$。若$G \sim \bar{G}$,不妨设同态为$\varphi$。则$G/Ker\varphi \cong \bar{G}$,所以$|G/Ker\varphi| = |\bar{G}| = n$。
  而$ G/Ker\varphi \leq G$,所以$|G/Ker\varphi| \bigm| |G|$,即$n \bigm| m$。

  再证$\Leftarrow$。若$n \bigm| m$,构造函数$\varphi:G \to \bar{G}, \forall a^p \in G, 0 \leq p < m, \varphi(a^p) = b^{q}, p \equiv q(\bmod\;n)$。
  $\forall a^p, a^q \in G, 0 \leq p, q < m, \varphi(a^p \circ a^q) = \varphi(a^{p + q}) = b^s, s \equiv p + q (\bmod\;n)$
  又有$ \varphi(a^p) * \varphi(a^q) = b^{s_1} * b^{s_2} = b^{s_1 + s_2}$,因为
  \begin{gather*}
    s_1 \equiv p (\bmod\;n) \\
    s_2 \equiv q (\bmod\;n) \\
    s_1 + s_2 \equiv p + q(\bmod\;n) \\
    \therefore s_1 + s_2 \equiv s(\bmod\;n) \\
    b^{s_1 + s_2} = b^s
  \end{gather*}
  所以$\varphi$是一个$G$到$\bar{G}$的同态。由于$n \bigm| m$,可知$n \leq m, \forall b^k \in \bar{G}, \varphi(a^k) = b^k$。
  所以$\varphi$是一个满射,$G \sim \bar{G}$。

  综上所述,命题得证。
\end{proof*}

\subsection{第二题}
\begin{proof*}
  不妨设$G = \{e, a, a^2, \ldots, a^{n - 1}\}, H = \{e, a^k, a^{2k}, \ldots, a^{n - k}\}$。
  则$|H| \bigm| |G|$,由第一题结论有$G \sim H$,设其同态为$\varphi:G \to \bar{G}, \forall a^p \in G, 0 \leq p < m, \varphi(a^p) = b^{q}, p \equiv q(\bmod\;n)$。
  则$Ker\varphi = \{a^{ik}\} = H$,所以$G \cong G / H$。由于$G$为循环群,所以$G / H$也是循环群。
\end{proof*}

\subsection{第三题}
1.构造函数$\varphi:G \to S_4, \bar{G} = \{\bar{e}, \bar{x}, \bar{y}, \bar{z}\}$,其中
\begin{align*}
  \bar{e} = \varphi(e) & = (1)        \\
  \bar{x} = \varphi(x) & = (1 2)(3 4) \\
  \bar{y} = \varphi(y) & = (1 3)(2 4) \\
  \bar{z} = \varphi(z) & = (1 4)(2 3)
\end{align*}

设$\bullet$为置换的合成,则可以得到表1
\begin{table}[H]
  \centering
  \begin{tabular}{c|cccc}
    \toprule
    $\bullet$ & $\bar{e}$ & $\bar{x}$ & $\bar{y}$ & $\bar{z}$ \\
    \midrule
    $\bar{e}$ & $\bar{e}$ & $\bar{x}$ & $\bar{y}$ & $\bar{z}$ \\
    \midrule
    $\bar{x}$ & $\bar{x}$ & $\bar{e}$ & $\bar{z}$ & $\bar{y}$ \\
    \midrule
    $\bar{y}$ & $\bar{y}$ & $\bar{z}$ & $\bar{e}$ & $\bar{x}$ \\
    \midrule
    $\bar{z}$ & $\bar{z}$ & $\bar{y}$ & $\bar{x}$ & $\bar{e}$ \\
    \bottomrule
  \end{tabular}
  \caption{$(\bar{G}, \bullet)$的乘法表}
\end{table}

观察两者的乘法表不难发现,$(G, \circ, e)$与$(\bar{G}, \bullet, \bar{e})$同构。

2.首先证明$(G_1, *)$为一个群。
\begin{proof*}
  显然$*$具备封闭性和交换律,且$\forall x \in G_1, e_1 * x = x, x * x = e_1$,
  即$e_1$为幺元,每个元素的逆元为其自身,只需证明$*$具有结合律。

  $\forall x, y \in G_1$
  \begin{gather*}
    x * (y * e) = x * y = (x * y) * e \\
    x * (e * y) = x * y = (x * e) * y \\
    e * (x * y) = x * y = (e * x) * y
  \end{gather*}
  而$a * (a * a) = a = (a * a) * a$,所以$*$具有结合律,所以$(G_1, *, e_1)$为一个群。
\end{proof*}
再证明$\varphi$为一个同态。
\begin{proof*}
  通过枚举有表2和表3
  \begin{table}[h]
    \begin{minipage}{0.48\linewidth}
      \centering
      \begin{tabular}{lcccc}
        \toprule
        \diagbox{p}{q} & e     & x     & y     & z     \\
        \midrule
        e              & $e_1$ & $e_1$ & a     & a     \\
        \midrule
        x              & $e_1$ & $e_1$ & a     & a     \\
        \midrule
        y              & a     & a     & $e_1$ & $e_1$ \\
        \midrule
        z              & a     & a     & $e_1$ & $e_1$ \\
        \bottomrule
      \end{tabular}
      \caption{$\varphi(p \circ q)$的值}
    \end{minipage}
    \hfill
    \begin{minipage}{0.48\linewidth}
      \centering
      \begin{tabular}{lcccc}
        \toprule
        \diagbox{p}{q} & e     & x     & y     & z     \\
        \midrule
        e              & $e_1$ & $e_1$ & a     & a     \\
        \midrule
        x              & $e_1$ & $e_1$ & a     & a     \\
        \midrule
        y              & a     & a     & $e_1$ & $e_1$ \\
        \midrule
        z              & a     & a     & $e_1$ & $e_1$ \\
        \bottomrule
      \end{tabular}
      \caption{$\varphi(p)*\varphi(q)$的值}
    \end{minipage}
  \end{table}

  所以$\forall p, q \in G, \varphi(p \circ q) = \varphi(p) * \varphi(q)$,$\varphi$为一个$G$到$G_1$的同态。
  \begin{gather*}
    \gamma = \{(e, \{e, x\}), (e, \{e, x\}), (y, \{y, z\}), (z, \{y, z\}) \} \\
    \bar{\varphi} = \{(\{e, x\}, e_1), (\{y, z\}, a) \} \\
    \varphi = \bar{\varphi} \circ \gamma
  \end{gather*}
\end{proof*}

\subsection{第四题}
\begin{proof*}
  显然$Z(\sqrt{2}) \subset R$,而由于$(R, +, \times)$是一个环,
  所以若$(Z(\sqrt{2}), + *)$为环,则$Z(\sqrt{2})$为$R$的一个子环,可知我们只需证明$\forall a, b \in Z(\sqrt{2}), ab, a - b \in Z(\sqrt{2})$。

  $\forall a, b \in Z(\sqrt{2}), \exists m_1, m_2, n_1, n_2 \in Z, a = m_1 + n_1\sqrt{2}, b = m_2 + n_2\sqrt{2}$
  \begin{align*}
    ab    & = (m_1 + n_1\sqrt{2}) \times (m_2 + n_2\sqrt{2}) \\
          & = (m_1m_2 + 2n_1n_2) + (m_1n_2 + m_2n_1)\sqrt{2} \in Z(\sqrt{2}) \\
    a - b & = (m_1 + n_1\sqrt{2}) - (m_2 + n_2\sqrt{2}) \\
          & = (m_1 - m_2) + (n_1 - n_2)\sqrt{2} \in Z(\sqrt{2})
  \end{align*}

  所以$(Z(\sqrt{2}), +, \times)$为一个环。
\end{proof*}

\subsection{第五题}
\begin{proof*}
  先证明$(Z(i), +)$是一个Abel群。
  $\forall a, b, c \in Z(i), \exists m_1, m_2, m_3, n_1, n_2, n_3 \in Z, a = m_1 + n_1i, b = m_2 + n_2i, c = m_3 + n_3i$
  \begin{align*}
    a + b &= (m_1 + n_1i) + (m_2 + n_2i) \\
          &= (m_1 + m_2) + (n_1 + n_2)i \in Z(i) \\
    b + a &= (m_2 + n_2i) + (m_1 + n_1i) \\
          &= (m_2 + m_1) + (n_2 + n_1)i \\
          &= (m_1 + m_2) + (n_1 + n_2)i \\
          &= a + b \\
    (a + b) + c &= ((m_1 + n_1i) + (m_2 + n_2i)) + (m_3 + n_3i) \\
          &= (m_1 + m_2 + m_3) + (n_1 + n_2 + n_3)i \\
          &= (m_1 + n_1i) + ((m_2 + n_2i) + (m_3 + n_3i)) \\
          &= a + (b + c) \\
    0 + a &= 0 + (m_1 + n_1i) \\
          &= m_1 + n_1i \\
          &= a \\
    (-m_1 - n_1i) + a &= (-m_1 - n_1i) + (m_1 + n_1i) \\
          &= 0
  \end{align*}
  所以$+$是$Z(i)$上的二元运算,满足交换律和结合律,且代数系存在幺元,每个元素存在左逆元。
  所以$(Z(i), +, 0)$是一个Abel群。

  再证明$(Z(i), \times)$是一个半群。
  $\forall a, b, c \in Z(i), \exists m_1, m_2, m_3, n_1, n_2, n_3 \in Z, a = m_1 + n_1i, b = m_2 + n_2i, c = m_3 + n_3i$
  \begin{align*}
    a \times b &= (m_1 + n_1i) \times (m_2 + n_2i) \\
          &= (m_1m_2 - n_1n_2) + (m_1n_2 + m_2n_1)i \in Z(i) \\
    (a \times b) \times c &= ((m_1 + n_1i) \times (m_2 + n_2i)) \times (m_3 + n_3i) \\
          &= ((m_1m_2 - n_1n_2) + (m_1n_2 + m_2n_1)i) \times (m_3 + n_3i) \\
          &= (m_1m_2m_3 - m_1n_2n_3 - n_1m_2n_3 - n_1n_2m_3) + (m_1m_2n_3 + m_1n_2m_3 + n_1m_2m_3 - n_1n_2n_3)i \\
    a \times (b \times c) &= (m_1 + n_1i) \times ((m_2 + n_2i) \times (m_3 + n_3i)) \\
          &= (m_1 + n_1i) \times ((m_2m_3 - n_2n_3) + (m_2n_3 + m_3n_2)i) \\
          &= (m_1m_2m_3 - m_1n_2n_3 - n_1m_2n_3 - n_1n_2m_3) + (m_1m_2n_3 + m_1n_2m_3 + n_1m_2m_3 - n_1n_2n_3)i \\
    (a \times b) \times c &= a \times (b \times c)
  \end{align*}
  所以$\times$是$Z(i)$上的二元运算,且满足结合律,所以$(Z(i), \times)$是一个半群。

  最后证明$\times$对$+$满足分配律。
  $\forall a, b, c \in Z(i), \exists m_1, m_2, m_3, n_1, n_2, n_3 \in Z, a = m_1 + n_1i, b = m_2 + n_2i, c = m_3 + n_3i$
  \begin{align*}
    a \times (b + c) &= (m_1m_2 + m_1m_3 - n_1n_2 - n_1n_3) + (n_1m_2 + n_1m_3 + m_1n_2 + m_1n_3)i \\
                &= (m_1m_2 - n_1n_2) + (n_1m_2 + m_1n_2)i + (m_1m_2 - n_1n_3) + (n_1m_3 + m_1n_3)i \\
                &= a \times b + a \times c \\
    (b + c) \times a &= (m_1m_2 + m_1m_3 - n_1n_2 - n_1n_3) + (n_1m_2 + n_1m_3 + m_1n_2 + m_1n_3)i \\
                &= b \times a + c \times a
  \end{align*}

  综上所述,$(Z(i), +, \times)$为一个环。
\end{proof*}

\subsection{第六题}
\begin{proof*}
  取$(1 + \sqrt[3]{2}) \in Q(\sqrt[3]{2})$,则
  $(1 + \sqrt[3]{2}) \times (1 + \sqrt[3]{2}) &= 1 + 2\sqrt[3]{2} + 2^\frac{2}{3} \notin Q(\sqrt[3]{2})$,
  所以$\times$对$Q(\sqrt[3]{2})$不封闭,$Q(\sqrt[3]{2})$对数的通常加法和乘法不构成一个环。
\end{proof*}

\subsection{第七题}
\begin{proof*}
  由于$(R, +, \times)$是一个域,$Q(\sqrt[3]{2}, \sqrt[3]{4}) \subset R$,$\times$具有交换律。
  所以若可以证明$Q(\sqrt[3]{2}, \sqrt[3]{4})$是$R$的一个子体,则命题得证。

  显然$|Q(\sqrt[3]{2}, \sqrt[3]{4})| \geq 2$。
  $\forall x, y \in Q(\sqrt[3]{2}, \sqrt[3]{4}), \exists a_1, a_2, b_1, b_2, c_1, c_2 \in Q
  x - y = (a_1 + b_1\sqrt[3]{2} + c_1\sqrt[3]{4}) - (a_2 + b_2\sqrt[3]{2} + c_2\sqrt[3]{4})
  = (a_1 - a_2) + (b_1 - b_2)\sqrt[3]{2} + (c_1 - c_2)\sqrt[3]{4} \in Q(\sqrt[3]{2}, \sqrt[3]{4})$,
  如果$x \neq 0, y \neq 0$,则$a_1, b_1, c_1$不同时为零,$a_2, b_2, c_2$不同时为零。
  则$ab^{-1} \in Q(\sqrt[3]{2}, \sqrt[3]{4})$。
  所以$Q(\sqrt[3]{2}, \sqrt[3]{4})$是$R$的一个子体。

  综上所述,$Q(\sqrt[3]{2}, \sqrt[3]{4})$是一个域。
\end{proof*}

\subsection{第八题}
\begin{proof*}
  若$e$不是单位元,则$e$不是右单位元,$\exists a \in R, ea = a, ae \neq a$,则$ae - a + e \neq e$。
  而$\forall b \in R, (ae - a + e)b = aeb - ab + b = b$,所以$ae - a + e$也是一个左单位元。
  这与$e$为唯一左单位元矛盾,所以$e$是单位元。
\end{proof*}

\subsection{第九题}
\begin{proof*}
  由于$a$,$b$,$ab - 1$有逆元,所以
  \begin{align*}
    (a - b^{-1})(b{(ab - 1)}^{-1}) &= ((a - b^{-1})b){(ab - 1)}^{-1} \\
                                   &=  (ab - 1){(ab - 1)}^{-1} \\
                                   &= 1 \\
    \therefore {(a - b^{-1})}^{-1} &= b{(ab - 1)}^{-1} \\
    ({(a - b^{-1})}^{-1} - a^{-1})(aba - a) &= (b{(ab - 1)}^{-1} - a^{-1})(ab - 1)a \\
                                   &=(b - a^{-1}(ab - 1))a \\
                                   &= ba - ba + 1 \\
                                   &= 1 \\
    \therefore {({(a - b^{-1})}^{-1} - a^{-1})}^{-1} &= (aba - a)
  \end{align*}

  命题得证。
\end{proof*}

\subsection{第十题}
\begin{proof*}
  不妨设$a$为$R$的一个零因子,则$\exists b \in R, b \neq 0, ab = 0$。
  若$a$存在逆元素$a^{-1}$,则
  \begin{gather*}
    ab = 0 \\
    a^{-1}ab = 0 \\
    b = 0
  \end{gather*}

  产生矛盾,所以$a$不存在逆元。
\end{proof*}

\subsection{第十一题}
\begin{proof*}
  由于$a, b$在交换环中,所以$ab = ba$,由环的性质有
  \begin{align*}
    {(a + b)}^{n} = \sum_{i = 1}^{n}\bigl( \begin{array}{c} n \\ i \end{array} \bigr) a^ib^{n-i}
  \end{align*}

  命题得证。
\end{proof*}

\subsection{第十二题}
\begin{proof*}
  显然$0$无左逆元。对于任一有左逆元的非零元素$a$,设其左逆元为$a_l$。
  若$a$无右逆元,则$aa_l \neq e, a_{l}aa_l \neq a_l, a_l \neq a_l$,产生矛盾,所以$a$有右逆元,进而有逆元。
\end{proof*}

\subsection{第十三题}
\begin{proof*}
  若$ab = ba$,则
  \begin{align*}
    a(-b) &= -(ab) \\
          &= -(ba) \\
          &= (-b)a \\
    a(-ab) &= -(aab) \\
          &= -(aba) \\
          &= (-ab)a \\
  \end{align*}
  所以$a$与$−b$,$a$与$−ab$可交换。

  若$a$与$b, c$可交换,即$ab = ba, ac = ca$,则
  \begin{align*}
    a(b + c) &= ab + ac \\
             &= ba + ca \\
             &= (b + c)a \\
    a(a + c) &= aa + ac \\
             &= aa + ca \\
             &= (a + c)a
  \end{align*}
  所以$a$与$b+c$,$a$与$a+c$可交换。
\end{proof*}

\end{document}
