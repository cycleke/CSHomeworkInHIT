\documentclass[11pt]{article}

\usepackage[fleqn]{amsmath}
\usepackage{amssymb}
\usepackage{amsthm}
\usepackage{babel}
\usepackage{bookmark}
\usepackage{booktabs}
\usepackage{capt-of}
\usepackage{colortbl}
\usepackage{dcolumn}
\usepackage{fancyhdr}
\usepackage[T1]{fontenc}
\usepackage{graphicx}
\usepackage{grffile}
\usepackage{hyperref}
\usepackage[utf8]{inputenc}
\usepackage{indentfirst}
\usepackage{longtable}
\usepackage{rotating}
\usepackage{textcomp}
\usepackage{tikz}
\usepackage[normalem]{ulem}
\usepackage{verbatim}
\usepackage{wrapfig}
\usepackage{xeCJK}

\usetikzlibrary{mindmap,trees}

\newcommand{\rmnum}[1]{\romannumeral #1}
\newcommand{\HRule}{\rule{\linewidth}{0.5mm}}
\newcommand{\Rmnum}[1]{\expandafter\@slowromancap\romannumeral #1@}
\setCJKmonofont{SimSun}
\setCJKmainfont[BoldFont=SimHei]{SimSun}
\setmainfont{Times New Roman}
\setlength{\parindent}{2em}

\graphicspath{{figures/}}
\allowdisplaybreaks
\renewcommand*{\qedsymbol}{[证毕]}


\title{近世代数作业1}
\author{cycleke}
\pagestyle{fancy}

\begin{document}
\maketitle
\tableofcontents\clearpage
\pagestyle{fancy}
\lfoot{}
\cfoot{\thepage}\rfoot{}
\setcounter{section}{0}
\setcounter{page}{1}
\clearpage

\section{课后习题}

\subsection{第3题}
\begin{proof}[证]
  因为$(S,\circ)$是一个半群,所以$\circ$符合交换律,即
  \begin{align*}
    \because (a \circ b) \circ x &= (a \circ b) \circ y \\
    \therefore a \circ (b \circ x) &= a \circ (b \circ y)
  \end{align*}
  又有a和b为左消去元,那么有:
  \begin{align*}
    &a \circ (b \circ x) = a \circ (b \circ y) \\
    \Rightarrow &b \circ x = b \circ y \ (\text{消去a}) \\
    \Rightarrow &x = y \ (\text{消去b})
  \end{align*}
  所以$a \circ b$为左消去元。
\end{proof}

\subsection{第4题}

\begin{proof}[1.证]
  首先证明$(M, \circ)$是一个半群,即$\circ$符合交换律。
  \begin{gather*}
    \quad \forall (x_1, x_2),(y_1, y_2),(z_1, z_2) \in M, \\
    \quad ((x_1, x_2) \circ (y_1, y_2)) \circ (z_1, z_2) \\
    = (x_1y_1 + 2x_2y_2, x_1y_2 + x_2y_1) \circ (z_1, z_2) \\
    = ((x_1y_1 + 2x_2y_2) \times z_1 + 2(x_1y_2 + x_2y_1) \times z_2, (x_1y_1 + 2x_2y_2) \times z_2 + (x_1y_2 + x_2y_1) \times z_1) \\
    = (x_1y_1z_1 + 2x_1y_2z_2 + 2x_2y_1z_2 + 2x_2y_2z_1, x_1y_1z_2 + x_1y_2z_1 + x_2y_1z_1 + 2x_2y_2z_2) \\
    \\
    \quad (x_1, x_2) \circ ((y_1, y_2) \circ (z_1, z_2)) \\
    = (x_1, x_2) \circ (y_1z_1 + 2y_2z_2, y_1z_2 + y_2z_1) \\
    = (x_1 \times (y_1z_1 + 2y_2z_2) + 2x_2 \times (y_1z_2 + y_2z_1), x_1 \times (y_1z_2 + y_2z_1) + x_2 \times (y_1z_1 + 2y_2z_2)) \\
    = (x_1y_1z_1 + 2x_1y_2z_2 + 2x_2y_1z_2 + 2x_2y_2z_1, x_1y_1z_2 + x_1y_2z_1 + x_2y_1z_1 + 2x_2y_2z_2) \\
    \therefore ((x_1, x_2) \circ (y_1, y_2)) \circ (z_1, z_2) = (x_1, x_2) \circ ((y_1, y_2) \circ (z_1, z_2))
  \end{gather*}
  \indent 再证明$(M, \circ)$是一个幺半群,即存在幺元。
  \begin{gather*}
    \quad \forall (x_1, x_2) \in M, \\
    \quad (1, 0) \circ (x_1, x_2)
    = (1 \times x_1 + 2 \times 0 \times x_2, 1 \times x_2 + 0 \times x_1)
    = (x_1, x_2) \\
    \\
    \quad (x_1, x_2) \circ (1, 0)
    = (x_1 \times 1 + 2 \times x_2 \times 0, x_1 \times 0 + x_2 \times 1)
    = (x_1, x_2) \\
    \therefore (1, 0) \text{是} (M, \circ) \text{的一个幺元}
  \end{gather*}
  \indent 综上所述,$(M, \circ)$是一个幺半群。
\end{proof}

\begin{proof}[2.证]
  \begin{gather*}
    \forall (x_1, x_2), (y_1, y_2), (z_1, z_2) \in M, s.t. (x_1, x_2) \neq (0, 0)\\
    \qquad (x_1, x_2) \circ (y_1, y_2) = (x_1, x_2) \circ (z_1, z_2) \\
    \Rightarrow \left\{
      \begin{array}{cc}
        x_1y_1 + 2x_2y_2 &= x_1z_1 + 2x_2z_2 \\
        x_1y_2 + x_2y_1 &= x_1z_2 + x_2z_1
      \end{array}
    \right.
    \Rightarrow \left\{
      \begin{array}{cr}
        x_1(y_1 - z_1) + 2x_2(y_2 - z_2) = 0 &\textcircled{1} \\
        x_2(y_1 - z_1) + x_1(y_2 - z_2) = 0 &\textcircled{2}
      \end{array}
    \right.\\
    \textcircled{1} \times x_2 - \textcircled{2} \times x_1 \Rightarrow (2x_2^2 - x_1^2) (y_2 - z_2) = 0 \\
    \textcircled{2} \times 2x_2 - \textcircled{1} \times x_1  \Rightarrow (2x_2^2 - x_1^2) (y_1 - z_1) = 0 \\
  \end{gather*}
  若 $2x_2^2 - x_1^2 = 0$,则有$x_1 = \pm \sqrt{2} x_2$,
  又因为$x_1, x_2 \in Z$,所以$x_1 = x_2 = 0$,这与$(x_1,x_2) \neq (0, 0)$矛盾。\\
  所以$2x_2^2 - x_1^2 \neq 0 \Rightarrow
  \left\{\begin{array}{c} y_2 - z_2 = 0 \\ y_1 - z_1 = 0 \end{array} \right.
  \Rightarrow \left\{\begin{array}{c} y_2 = z_2 \\ y_1 = z_1  \end{array} \right.
  \Rightarrow (y_1, y_2) = (z_1, z_2)$,所以$(x_1, x_2)$为左消去元。
\end{proof}

\begin{proof}[3.证]
  \begin{align*}
    (x_1, x_2) \circ (y_1, y_2) = (x_1y_1 + 2x_2y_2, x_1y_2 + x_2y_1) = (y_1, y_2) \circ (x_1, x_2)
  \end{align*}
  所以$\circ$符合交换律。
\end{proof}

\subsection{第5题}
\begin{proof}[证]
  有错误。步骤$x^{2(n-k)}x^k = x^{n-k}x^k \Rightarrow x^{2(n-k)} = x^{n-k}$成立的条件为$x^k$为半群$(S, \circ)$的右消去元,
  但是没证明$x^k$为右消去元,所以有错。
\end{proof}

\subsection{第6题}
\begin{proof}[证]
  设有限半群$(S, \circ),|S| = n$。\\
  若$\nexists a \in S, a \circ a = a$,则
  $\exists 2 \le m \le n, a_1 \circ a_1 = a_2, a_2 \circ a_2 = a_3, \cdots , a_m \circ a_m = a_1$,
  进而有:
  \begin{align*}
    a_1 \circ (a_1 \circ a_2 \circ \cdots \circ a_m) &= (a_1 \circ a_1) \circ (a_2 \circ \cdots \circ a_m) \\
                                &= a_2 \circ (a_2 \circ \cdots \circ a_m) \\
                                & \ldots \\
                                &= a_m \circ a_m \\
                                &= a_1
  \end{align*}
  设$ A = a_1 \circ a_2\circ \cdots \circ a_m$,因为$(S, \circ)$为半群,所以$A \in S$,进而有
  \begin{align*}
    &\quad a_1 \circ A = a_1 \circ a_1 \circ a_2\circ \cdots \circ a_m = a_1 \\
    &\Rightarrow (a_1 \circ a_1) \circ A = a_1 \circ a_1 \\
    &\Rightarrow a_2 \circ A = a_2 \\
    &\ldots \\
    &\Rightarrow a_m \circ A = a_m \\
    &\Rightarrow a_{m - 1} \circ a_m \circ A = a_{m -1} \circ a_m \circ A \\
    &\ldots \\
    &\Rightarrow a_1 \circ a_2\circ \cdots \circ a_m \circ A = a_1 \circ a_2\circ \cdots \circ a_m \\
    &\Rightarrow A \circ A = A
  \end{align*}
  这与$\nexists a \in S, a \circ a = a$矛盾,所以$\exists a \in S, a \circ a = a$,命题得证。
\end{proof}

\subsection{第7题}
\begin{proof}[证]
  \begin{align*}
    &\forall a, b, c \in M \\
    &\quad (a * b) * c \\
    &= (a \circ m \circ b) * c \\
    &= (a \circ m \circ b) \circ m \circ c \\
    &= a \circ m \circ (b \circ m \circ c) \\
    &= a * (b * c)
  \end{align*}
  所以$(M, *)$是一个半群。\\
  若$(M, *)$为一个幺半群,设其幺元为$e'$,则有
  \begin{gather*}
    \forall a \in M, a * e' = a \Rightarrow a \circ m \circ e' = a \Rightarrow m \circ e' = e
  \end{gather*}
  所以当m在$(M, \circ)$中存在逆元素$m^{-1}$时,$(M, *, e^{-1})$为幺半群。
\end{proof}

\subsection{第9题}
\begin{proof}[证]
  对$\forall A, B, C \in 2^S$,有
  \begin{gather*}
    (A \Delta B) \Delta C = (A \bigcap B^C \bigcap C^C) \bigcup (A^C \bigcap B \bigcap C^C) \bigcup (A^C \bigcap B^C \bigcap C) = A \Delta (B \Delta C)
  \end{gather*}
  所以$(2^S, \Delta)$是一个半群。\\
  因为$\forall A \in 2^S, A \Delta \emptyset = \emptyset \Delta A = A$,所以$(2^S, \Delta, \emptyset)$为幺半群。\\
  因为$\forall A \in 2^S, A \Delta A = \emptyset$,所以$A^{-1} = A$,所以$(2^S, \Delta)$为群。
\end{proof}

\section{思考题}
\subsection{思考题1}

设$Z_n = \{[0], [1], \cdots, [n - 1]\}$是整数集合 Z 上在模 n 的同余关系之下的等价类之集合。
在$Z_n$上定义乘法“$\times$”,$\forall [i], [j] \in Z_n, [i] \times [j] = [i \times j]$。\\
现证明$\times$为$Z_n$的一个二元运算:
\begin{proof}[证]
  \begin{gather*}
    \forall k \in [i], l \in [j] \\
    [k] = [i], [l] = [j], [k] \times [l] = [i] \times [j] \\
    [k] \times [l] = [k \times l], [i] \times [j] = [i \times j]
  \end{gather*}
  所以只需证$[k \times l] = [i \times j]$。
  由于$n | (k - i), n | (l - j)$,所以不妨设$k = i + an, l = j + bn, a,b \in Z$。
  所以$k \times l - i \times j = n (abn + aj + bi) \Rightarrow n | (k \times l - i \times j) \Rightarrow [k \times l] = [i \times j]$。
\end{proof}
又因为$\forall [i], [j], [k] \in Z_n$,
\begin{align*}
  ([i] \times [j]) \times [k] &= [i \times j \times k] \\
  [i] \times ([j] \times [k]) &= [i \times j \times k] \\
  \therefore ([i] \times [j]) \times [k] &= [i] \times ([j] \times [k])
\end{align*}
所以$(Z_n, \times)$是一个半群。
其中显然$(Z_1, \times)$有一个单位元$[0]$。\\
若要构造具有$n(n \geq 2)$ 个左(或右)单位元的半群,则有如下构造方法:\\
设
\begin{gather*}
  S = \{ \left(
    \begin{array}{cc}
      a & b \\
      0 & 0
    \end{array}
  \right) | a, b \in Z_n \} \\
\end{gather*}
在$S$上定义乘法$\times$,规则使用一般矩阵的乘法,不过将自然数的乘法变为$Z_n$上的乘法。
不难得出$\forall d \in Z_n, \left(
  \begin{array}{cc}
    [1] & d \\
    0 & 0
  \end{array}
\right)$是$(S, \times)$的左单元元素。又因为$|Z_n| = n$,所以$(S, \times)$有n个左单元元素。

同理可以构造\begin{gather*}
  S' = \{ \left(
    \begin{array}{cc}
      a & 0 \\
      b & 0
    \end{array}
  \right) | a, b \in Z_n \} \\
\end{gather*}
$(S', \times)$有n个右单元元素。

\end{document}
