% !Tex Program = xelatex
% -*-coding: utf-8 -*-
\documentclass[12pt,onecolumn]{article}

% 中文
\usepackage[BoldFont,SlantFont]{xeCJK}
\xeCJKsetemboldenfactor{1}%只对随后定义的CJK字体有效
\setCJKfamilyfont{hei}{SimHei}
\xeCJKsetemboldenfactor{4}
\setCJKfamilyfont{song}{SimSun}
\xeCJKsetemboldenfactor{4}
\setCJKfamilyfont{fs}{FangSong}
\setCJKfamilyfont{kai}{KaiTi}
\setCJKfamilyfont{li}{LiSu}
\setCJKfamilyfont{xw}{STXinwei}
\setCJKmainfont{SimSun}

\newcommand{\hei}{\CJKfamily{hei}}      % 黑体
\newcommand{\song}{\CJKfamily{song}}    % 宋体   (Windows 自带simsun.ttf)
\newcommand{\fs}{\CJKfamily{fs}}        % 仿宋体 (Windows 自带simfs.ttf)
\newcommand{\kai}{\CJKfamily{kai}}      % 楷体   (Windows 自带simkai.ttf)
\newcommand{\li}{\CJKfamily{li}}        % 隶书   (Windows自带simli.ttf)
\newcommand{\xw}{\CJKfamily{xw}}        % 隶书   (Windows自带simli.ttf)

% \AmSTeX\ 宏包,用来排出更加漂亮的公式。
\usepackage{amsmath}
% 定理类环境宏包,其中 \pkg{amsmath} 选项用来兼容 \AmSTeX\ 的宏包
\usepackage[amsmath,thmmarks,hyperref]{ntheorem}
\usepackage{amssymb}
% 添加字体
\usepackage[defaultsups]{newtxtext}
\usepackage{newtxmath}
\usepackage{courier}
% 图形支持宏包
\usepackage{graphicx}
% 插入pdf
\usepackage{pdfpages}
\includepdfset{fitpaper=true}
% 更好的列表环境。
\usepackage{enumitem}       %使用enumitem宏包,改变列表项的格式
\usepackage{environ}
% 禁止 \LaTeX 自动调整多余的页面底部空白,并保持脚注仍然在底部。
% 脚注按页编号。
\usepackage[bottom,perpage,hang]{footmisc}
\raggedbottom
% 脚注格式。
\usepackage{pifont}
% 表格控制
\usepackage{longtable}
\usepackage{booktabs}
% 参考文献引用宏包
\usepackage[sort&compress]{natbib}
% 生成有书签的 pdf 及其开关,请结合 gbk2uni 避免书签乱码。
\usepackage{hyperref}
\hypersetup{
  CJKbookmarks=true,
  linktoc=all,
  bookmarksnumbered=true,
  bookmarksopen=true,
  bookmarksopenlevel=1,
  breaklinks=true,
  colorlinks=false,
  plainpages=false,
  pdfborder=0 0 0}
% 设置 url 样式,与上下文一致
\urlstyle{same}
% 版芯设置
\usepackage{geometry}
\geometry{
  centering,
  text={150true mm,236true mm},
  left=30true mm,
  head=5true mm,
  headsep=2true mm,
  footskip=0true mm,
  foot=5.2true mm
}
% 利用 \pkg{fancyhdr} 设置页眉页脚。
\usepackage{fancyhdr}
% 其他包,表格、数学符号包
\usepackage{tabularx}
\usepackage{varwidth}
% 此处changepage环境用来控制索引页面的左右边距,规范中给出的示例的边距要大于正文。
\usepackage{changepage}
% 多栏结构在文中用begin{multicols}{2}end{multicols}
\usepackage{multicol,multienum}
% 允许上一个section的浮动图形出现在下一个section的开始部分,还提供\FloatBarrier命
% 令,使所有未处理的浮动图形立即被处理
\usepackage[below]{placeins}
% 支持子图 %centerlast 设置最后一行是否居中
\usepackage{subfigure}
% 支持双语标题
\usepackage[subfigure]{ccaption}
% 根据我工规定,正文小四号 (12bp) 字,行距为固定值3--4mm。
\renewcommand\normalsize{%
  % \@setfontsize\normalsize{12bp}{\ifhit@glue 20.50398bp \@plus 2.83465bp \@minus 0bp\else 20.50398bp\fi}%
  \abovedisplayskip=8pt
  \abovedisplayshortskip=8pt
  \belowdisplayskip=\abovedisplayskip
  \belowdisplayshortskip=\abovedisplayshortskip}
% 根据习惯定义字号。用法:\cs{hit@def@fontsize}\marg{字号名称}\marg{磅数}避免了
% 字号选择和行距的紧耦合。所有字号定义时为单倍行距,并提供选项指定行距倍数。
\def\hit@def@fontsize#1#2{%
  \expandafter\newcommand\csname #1\endcsname[1][1.3]{%
    \fontsize{#2}{##1\dimexpr #2}\selectfont}}
\hit@def@fontsize{dachu}{58bp}
\hit@def@fontsize{chuhao}{42bp}
\hit@def@fontsize{xiaochu}{36bp}
\hit@def@fontsize{yihao}{26bp}
\hit@def@fontsize{xiaoyi}{24bp}
\hit@def@fontsize{erhao}{22bp}
\hit@def@fontsize{xiaoer}{18bp}
\hit@def@fontsize{sanhao}{16bp}
\hit@def@fontsize{xiaosan}{15bp}
\hit@def@fontsize{sihao}{14bp}
\hit@def@fontsize{banxiaosi}{13bp}
\hit@def@fontsize{xiaosi}{12bp}
\hit@def@fontsize{dawu}{11bp}
\hit@def@fontsize{wuhao}{10.5bp}
\hit@def@fontsize{xiaowu}{9bp}
\hit@def@fontsize{liuhao}{7.5bp}
\hit@def@fontsize{xiaoliu}{6.5bp}
\hit@def@fontsize{qihao}{5.5bp}
\hit@def@fontsize{bahao}{5bp}
% 利用 \pkg{enumitem} 命令调整默认列表环境间的距离,以符合中文习惯。
\setlist{nosep}
% 允许太长的公式断行、分页等。
\allowdisplaybreaks[4]
\predisplaypenalty=0  %公式之前可以换页,公式出现在页面顶部
\postdisplaypenalty=0
% 公式编号设置
\renewcommand{\theequation}{\arabic{section}.\arabic{equation}}
% 定理标题使用黑体,正文使用宋体,冒号隔开。
\theorembodyfont{\normalfont}
\theoremheaderfont{\normalfont\hei}
\theoremsymbol{\ensuremath{\square}}
\newtheorem*{proof}{证明}
\theoremstyle{plain}
\theoremsymbol{}
\theoremseparator{}
\newtheorem{assumption}{假设}[section]
\newtheorem{definition}{定义}[section]
\newtheorem{proposition}{命题}[section]
\newtheorem{lemma}{引理}[section]
\newtheorem{theorem}{定理}[section]
\newtheorem{axiom}{公理}[section]
\newtheorem{corollary}{推论}[section]
\newtheorem{exercise}{练习}[section]
\newtheorem{example}{例}[section]
\newtheorem{remark}{注释}[section]
\newtheorem{problem}{问题}[section]
\newtheorem{conjecture}{猜想}[section]
% 各种单位
\usepackage{siunitx}
\sisetup{group-minimum-digits=4, group-separator= \hspace{0.25em}}
\sisetup{detect-weight,detect-mode,detect-family}
% 处理数学公式中的黑斜体的宏包
\usepackage{bm}
% 不同于 \mathcal \mathfrak 之类的英文花体字体
\usepackage{mathrsfs}
% 支持彩色
\usepackage{xcolor}
\definecolor{colorzero}{rgb}{0, 0, 0}
\definecolor{colorone}{rgb}{1, 0, 0}
\definecolor{colortwo}{rgb}{0, 0, 1}
\definecolor{colorthree}{rgb}{0, 1, 0}
% 图形和表格的控制旋转
\usepackage{rotating}
% 算法的宏包,注意宏包兼容性,先后顺序为float、hyperref、algorithm(2e),否则无法
% 生成算法列表。
\usepackage[algoruled,linesnumbered]{algorithm2e}
% 排版源码所使用的环境。
\usepackage{listings}
\lstset{
  breaklines  = true,
  captionpos  = b,
  tabsize     = 2,
  numbers     = left,
  columns     = flexible,
  keepspaces  = true,
  % commentstyle = \color[RGB]{0,128,0},
  % keywordstyle = \color[RGB]{0,0,255},
  basicstyle   = \small\ttfamily,
  rulesepcolor = \color{red!20!green!20!blue!20},
  showstringspaces = false,
}

% 作图
\usepackage{tikz}

% 首行缩进
\usepackage{indentfirst}
\setlength{\parindent}{2em}

\usepackage{float}
\usepackage{diagbox}
\usepackage{setspace}
\usepackage{zhnumber}

\graphicspath{{figures/}}

\pagestyle{fancy}
\fancyhead[L]{\song\xiaowu[0]{cycleke}}
\fancyhead[C]{\song\xiaowu[0]{哈尔滨工业大学}}
\fancyhead[R]{\song\xiaowu[0]{近世代数(2020春)课程报告}}
\fancyfoot[C]{\xiaowu-~\thepage~-}

\renewcommand{\today}{\number\year{年}\number\month{月}\number\day{日}}
\renewcommand{\figurename}{图}
\renewcommand{\tablename}{表}

\renewcommand{\baselinestretch}{1.25}

\begin{titlepage}
  \centering
  \hei\LARGE{}
  \vspace{3\baselineskip}

  近世代数(2020春)

  课程报告

  \vspace{16\baselineskip}

  \begin{flushleft}
    \hspace{2cm}班级:\underline{xxxxxx}

    \hspace{2cm}学号:\underline{xxxxxxxxxx}

    \hspace{2cm}姓名:\underline{cycleke}
  \end{flushleft}
\end{titlepage}

\begin{document}
\clearpage

\begin{table*}[h]
 \vspace{2\baselineskip}
 \xiaosi
 \begin{tabular}{|c|c|c|c|c|c|c|}
 \hline
 题号 & 一 & 二 & 三 & 四 & 五 & 总分 \\
 \hline
 得分 &    &    &    &    &    &      \\
 \hline
 \end{tabular}
\end{table*}

\section{阐述一下半群和群的关系,并举例说明如何由半群和群来构造一个有限环。(20分)}

\subsection{半群和群}
半群和群都是用于表示两类具有特别性质的代数系。
半群的定义如下:
\begin{definition}{(半群)}
 设“$\circ$”是非空集合$S$上的一个二元代数运算,称为乘法。
 如果$\forall a, b, c \in S$,有$(a \circ b) \circ c = a \circ (b \circ c)$,
 则称集合$S$对乘法$\circ$形成一个半群(semigroup),并记为$(S, \circ)$。
\end{definition}

而在半群中有一种特别的半群:幺半群。
\begin{definition}{(幺半群)}
 有单位元素的半群$(S, \circ)$称为独异点(monoid)或称为幺半群。
\end{definition}
即如果$(S, \circ)$是一个幺半群,那么$\exists e \in S$,使得$\forall a \in S, a \circ e = e \circ a = a$。

此时我们定义一个概念:逆元素。
\begin{definition}{(逆元素)}
 设$(S, \circ, e)$是一个幺半群。
 元素$a \in S$称为有逆元素,如果存在$b \in S$使$a \circ b = b \circ a = e$,此时$b$叫做$a$的逆元素。
\end{definition}
而群就是一种更加特别的幺半群:
\begin{definition}{(群)}
 对于幺半群$(S, \circ, e)$,如果其中中的每个元素$a \in S$,都存在一个逆元素$b \in S$,
 那么我们称它为一个群。
\end{definition}

总的来说,半群就是一个代数系$(S, \circ)$,乘法“$\circ$”在集合$S$上满足封闭性和结合律。
群则是一类特殊的半群,它的乘法“$\circ$”在集合$S$上不仅满足封闭性和结合律,
而且存在单位元素$e$,每个元素都存在一个逆元素。

\subsection{构造有限环}
环是一种更复杂的代数系。
\begin{definition}{(环)}
 设$R$是一个非空集合,$R$中有两个代数远算,
 一个叫做加法并用加号“$+$”表示,另一个叫做乘法并用“$\circ$”表示。
 如果
 \begin{enumerate}[fullwidth,itemindent=\parindent,label=(\arabic*)]
 \item $(R, +)$是一个 Abel 群;
 \item $(R, \circ)$是一个半群;
 \item 乘法对加法满足左、右分配律:$\forall a, b, c \in R$有
 \begin{align*}
 a \circ (b + c) & = (a \circ b) + (a \circ c) \\
 (b + c) \circ a & = (b \circ a) + (c \circ a)
 \end{align*}
 \end{enumerate}
 则称代数系$(R, +, \circ)$为一个环。
\end{definition}

所以我们可以考虑
群$G$,其集合为模5剩余类$Z_5$,“乘法”为模意义下的加法$\oplus, [i] \oplus [j] = [i + j]$;
半群$S$,其集合同为模5剩余类$Z_5$,“乘法”为模意义下的乘法$\otimes, [i] \otimes [j] = [i \times j]$。

容易证明(事实上在课本中已经证明了),$G$为一个 Abel 群,$S$为一个半群,
且模意义下的乘法$\otimes$对模意义下的加法$\oplus$满足分配律。
又因为$|Z_5| = 5$,所以$(Z_5, \oplus, \otimes)$是一个有限环。

\section{举例说明群的同构 Cayley 定理的意义。你认为在研究两个代数系统之间的关系时候,该定理有什么局限性没有?如果有,你有什么解决方案?(20分)}

\subsection{Cayley 定理的意义}
我们首先介绍变换群的概念。

\begin{definition}{(对称群)}
 设$S$是一个非空集合。从$S$到$S$的所有一一对应之集记为$Sym(S)$,
 则$Sym(S)$对映射的合成构成一个群,称为$S$上的对称群。n
\end{definition}

\begin{definition}{(变换群)}
 $Sym(S)$的任一子群称为$S$上的一个变换群。
\end{definition}

\begin{theorem}{(群的 Cayley 同构定理)}
 任何一个群都同构于某个变换群。
\end{theorem}

更为具体地,每一个群$G$都同构于变换群$L(G) = \{ f_a \mid f_a : G \to G, \forall x \in G, f_a(x) = a \circ x, a \in G \}$。
特别地,对于有限群,$L(G)$是一个置换群,即
\[
 L(G) =
 \left\{
 \left(
 \begin{array}{cccc}
 a_1    & a_2    & \ldots & a_n    \\
 a_ia_1 & a_ia_2 & \ldots & a_ia_n
 \end{array}
 \right)
 \bigg| a_i \in G
 \right\}
\]

利用 Cayley 定理,我们可以为任何一个抽象的群找到一个对应的变换群模型。
相比与原本的抽象群,我们找到的变换群比较具体,
其元素是一一对应的,代数运算是一一对应的合成,
不但具体而且容易计算。
因此,用变换群给出抽象群的例子是方便和合理的。
经验告诉我们,相比与抽象,具体的例子总是更容易理解和分析的。

\subsection{Cayley 定理的局限性}
Cayley 定理要求我们研究的对象必须是一个群。
对于一个抽象的代数系统,
如果它不是一个群,那么我们就无法应用 Cayley 定理,
那么就不得不又去研究那个抽象的代数系统。

\subsection{对于局限性的解决方案}
为了解决 Cayley 定理的局限性,我们需要对 Cayley 定理进行一定程度的拓展。

\begin{theorem}
 任何一个半群都同构于某个对称群的子集。
\end{theorem}

\begin{proof}
 设$(S, \circ)$为一个半群。
 构造$L = \{f_a \mid f_a : S \to S, \forall x \in S, f_a(x) = a \circ x, a \in G\}$。

 首先证明$L$对于映射的合成构成一个半群。
 事实上,$\forall f_a, f_b \in L$,$f_a \bullet f_b(x) = f_a(f_b(x)) = a \circ b \circ x = f_{a \circ b} (x)$,
 所以$f_{a \circ b} = f_a \bullet f_b$,$\bullet$在$L$上封闭。
 又因为映射的合成满足结合律,所以$(L, \bullet)$构成一个半群。

 令映射$\varphi : S \to L, \varphi(a) = f_a, \forall a \in G$,不难发现$\varphi$是一一对应且
 $\forall a, b \in G, \varphi(a \circ b) = f_{a \circ b} = f_a \bullet f_b = \varphi(a) \bullet \varphi(b)$。
 因此,$\varphi$是一个$S$到$L$的同构。
\end{proof}

这样我们就可以将一个抽象的半群具体化为具体的半群。
而对于其他的代数系统,我们可以亦根据其性质来构造对应的具体的代数系。

\section{结合实例给出判定一个子群是否为正规子群的方法,并说明在代数系统研究中正规子群有什么重要应用?(20分)}

\subsection{判定正规子群}

\begin{definition}
 设$H$是群$G$的子群。
 如果$\forall a \in G$,有$aH = Ha$,
 则称$H$是$G$的正规子群,
 记为$H \triangleleft G$。
\end{definition}

正规子群是左右陪集相等的子群,
我们可以直接使用定义来判定一个子群是否为正规子群。
如对于群$(Z_5, \oplus)$,其子群$H_1 = (\{[0]\}, \oplus)$和$H_2 = (\{[0], [1]\}, \oplus)$,
容易证明$\forall a \in Z_5, \text{设}a = [x], aH_1 = \{a\} = H_1a$,
$aH_2 = \{[x], [x + 1]\} = H_2a$。
所以$H_1, H_2$均为群$(Z_5, \oplus)$的正规子群。

使用定义判定很直接,
但是使用时不够灵活,证明较繁琐。
我们平时还可以使用正规子群的充要条件(定理\ref{theo:normal_subgroup})来判定正规子群。

\begin{theorem}
 \label{theo:normal_subgroup}
 设$H$是群$G$的一个子群,则下面三个命题等价:
 \begin{enumerate}[fullwidth,itemindent=\parindent,label=(\arabic*)]
 \item $H$是$G$的正规子群
 \item $\forall a \in G, aHa^{-1} = H$
 \item $\forall a \in G, aHa^{-1} \subseteq H$
 \end{enumerate}
\end{theorem}

容易看出,第三个命题是较弱的,
故我们一般采用命题三来判定正规子群。
一个例子是我们可以使用命题三来证明换位子群是正规子群。

\begin{theorem}
 群$G$的换位子群$H$是$G$的正规子群。
\end{theorem}
\begin{proof}
 因为$H$是$G$的换位子群,所以$\forall h \in H, g \in G$,有
 $ghg^{-1}h^{-1} \in H$,于是$ghg^{-1} = (ghg^{-1}h^{-1})h \in H$。
 所以$\forall g \in G, gHg^{-1} \subseteq H$,$H$是$G$的正规子群。
\end{proof}

而使用命题二我们亦可以判定一个正规子群。
在群$(S_3, \circ)$的子群中,
$H = \{I, (1~2~3), (1~3~2)\}$就构成了一个正规子群,
判定如下:
\begin{gather*}
 IHI              = H \\
 (1~2)H(1~2)      = H \\
 (1~3)H(1~3)      = H \\
 (2~3)H(2~3)      = H \\
 (1~2~3)H(1~3~2)  = H \\
 (1~3~2)H(1~2~3)  = H
\end{gather*}


除此以外,我们还可以使用其他的定理来判定一个子群是否为正规子群。

\begin{theorem}
 群$G$的子群$H$是$G$的正规子群当且仅当对$G$的任一内自同构$\varphi$有$\varphi(H) = H$。
\end{theorem}

不难发现,该定理的判定条件与定理\ref{theo:normal_subgroup}中的命题二是等价的。

\subsection{正规子群的应用}
正规子群在群论中占有重要的地位。
正规子群之所以重要是它的陪集构成的集群对群子集形成群。

\begin{theorem}
 设$H$是$G$的正规子群,$H$的所有左陪集构成的集族$S_l$对群子集乘法形成一个群。
\end{theorem}

\begin{definition}
 群$G$的正规子群$H$的所有左陪集构成的集族,对群子集乘法构成的群称为$G$对$H$的商群,记为$G / H$。
\end{definition}

我们可以看到,每个正规子群,对应了一个商群。
商群的乘法是借助与$G$的乘法定义的,
而商群$G / H$与群$G$有密切的联系。
因此,对$G / H$的研究将有助于对$G$的性质的深刻认识。
因此,正规子群在群论中占有重要的地位。

此外,利用正规子群$H$对应的商群$G / H$,
我们可以定义$G$上的一个等价关系。
这个等价关系是这样的:
$\forall a, b \in G$,
$a \cong b$当且仅当$ab^{-1} \in H$。
而且这个等价关系是同余关系,即
若$a \cong b, a' \cong b'$,则$a \circ a' \cong b \circ b'$。
此外正规子群在群的同态基本定理(定理\ref{theo:homomorphism})中也有应用。

\begin{theorem}
 设$H$是群$G$的子群,则$H$是正规子群的充分必要条件是
 $G$上的由$H$确定的等价关系“$\cong$”:
 \[
 a \cong b \text{当且仅当} ab^{-1} \in H
 \]
\end{theorem}

\section{通过实例来简述群的同态基本定理及其意义,并结合你的例子指出其中同态核的意义。(30分)}

\subsection{群的同态基本定理及其意义}

\begin{definition}
 设$\varphi$是群$(G, \circ)$到群$(\bar{G}, \bullet)$的满同态,
 $\bar{e}$是$\bar{G}$的单位元,
 则$G$的正规子群$\varphi^{-1}(\bar{e})$称为同态$\varphi$的核,记为$Ker~\varphi$。
 $\varphi(G)$称为在$\varphi$下$G$的同态象。
\end{definition}

\begin{theorem}{(群的同态基本定理)}
 \label{theo:homomorphism}
 设$\varphi$是群$SG$到群$\bar{G}$的满同态,$E = Ker~ \varphi$,则
 \[
 G / E \cong \bar{G}
 \]
\end{theorem}

群的同态基本定理的意义在于,
设$G$的性质已经比较清楚,
而群$\bar{G}$的性质的尚不太清楚,
那么基于定理\ref{theo:homomorphism},
如果能建立一个从$G$到$\bar{G}$的满同态$\varphi$,
则$\bar{G} \cong G / Ker~\varphi$,
从而$\bar{G}$与$G / Ker~ \varphi$的性质完全一样。
而群$G / Ker~\varphi$是$G$的正规子群 $Ker~\varphi$的陪集,
是$G$的一个特殊的等价类划分,其乘法是借由$G$的乘法定义的。
因此,$G / Ker~\varphi$的性质容易从$G$的一些性质得到。
也就是说,定理\ref{theo:homomorphism}为我们提供了
一种从已知群推导另一个群的性质的方法。

例如,我们已知群$G = (Z_6, \oplus)$这个群的性质,
如果我们要研究群$\bar{G} = (\{0, \frac{2\pi}{3}, \frac{4\pi}{3}\}, \circ)$,
$\bar{G}$表示的三个用弧度表示的角,其的乘法为逆时针旋转。

不难得出,$G$的单位元为$[0]$,$\bar{G}$的单位元为$0$。
$G$到$\bar{G}$的满同态$\varphi: G \to \bar{G},
\varphi([x]) = \frac{2\pi (x\mod 3)}{3}$。
所以$Ker~\varphi = \{[0], [3]\}$,根据定理\ref{theo:homomorphism}有
\[
  \bar{G} \cong G / Ker~\varphi = \{\{[0], [3]\}, \{[1], [4]\}, \{[2], [5]\}\}
\]

故有结论$\bar{G}$是一个三阶循环群。

\subsection{同态核的意义}

$Ker~\varphi$不仅是$G$的一个子群,而且是一个正规子群。
因此,商群$G / Ker~G$变得具有意义。
在上面的例子中,
我们是依据同态核$Ker~\varphi$来将群$G$来进行划分为
$\{\{[0], [3]\}, \{[1], [4]\}, \{[2], [5]\}\}$。
我们将模6意义下差(或者说$G$中的商)在$Ker~\varphi$中的
元素划分到同一个集合。


\section{谈一下抽象代数系统的学习给你带来的体会,以及你对此门课程今后的建议。(10分)}
在以前,我学习的都是具体的代数系统,如
实数$\mathbb{R}$上对四则运算法构成的算数系统和
包含了集合论运算(如并集、交集、补集)的集合代数。
而抽象代数系统是对于一类代数系统的抽象,而不是某一个特定的代数系统。
它不关注某个特定代数系统本身的性质,
而是关注一类具有相似点的代数系统共有的性质。
相比与原本具体的代数系统,
抽象代数系统的抽象层次提升了许多,
对应地,学习和理解的难度也提升了许多。

抽象代数系统的学习提高了我的抽象能力和理解能力。
在计算机理论和应用中,
抽象是一项十分重要的能力,
如计算建模就是将实际问题抽象为一个模型来研究。
抽象代数系统的定理更加抽象,
同时也更加一般,可以拓展到更多的代数系统中。
利用抽象代数系统,我们可以实现代数系统的公理化,
严谨地研究更为一般的代数系统,得到更为一般的理论。
例如,各类置换群的结论可被视为“抽象群”的一般概念有关之一般性定理的特例。

由于本学期是在线上授课,
我感觉上课过程中关于证明过程和证明思路的教学不足,
个人感觉有点遗憾,
希望在日后的教学中可以强化这一方面。

\end{document}
