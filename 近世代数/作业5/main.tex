% !Tex Program = xelatex
% -*-coding: utf-8 -*-
\documentclass[12pt,onecolumn]{article}

% 中文
\usepackage[BoldFont,SlantFont]{xeCJK}
\xeCJKsetemboldenfactor{1}%只对随后定义的CJK字体有效
\setCJKfamilyfont{hei}{SimHei}
\xeCJKsetemboldenfactor{4}
\setCJKfamilyfont{song}{SimSun}
\xeCJKsetemboldenfactor{4}
\setCJKfamilyfont{fs}{FangSong}
\setCJKfamilyfont{kai}{KaiTi}
\setCJKfamilyfont{li}{LiSu}
\setCJKfamilyfont{xw}{STXinwei}
\setCJKmainfont{SimSun}

\newcommand{\hei}{\CJKfamily{hei}}      % 黑体
\newcommand{\song}{\CJKfamily{song}}    % 宋体   (Windows 自带simsun.ttf)
\newcommand{\fs}{\CJKfamily{fs}}        % 仿宋体 (Windows 自带simfs.ttf)
\newcommand{\kai}{\CJKfamily{kai}}      % 楷体   (Windows 自带simkai.ttf)
\newcommand{\li}{\CJKfamily{li}}        % 隶书   (Windows自带simli.ttf)
\newcommand{\xw}{\CJKfamily{xw}}        % 隶书   (Windows自带simli.ttf)

% \AmSTeX\ 宏包,用来排出更加漂亮的公式。
\usepackage{amsmath}
% 定理类环境宏包,其中 \pkg{amsmath} 选项用来兼容 \AmSTeX\ 的宏包
\usepackage[amsmath,thmmarks,hyperref]{ntheorem}
\usepackage{amssymb}
% 添加字体
\usepackage[defaultsups]{newtxtext}
\usepackage{newtxmath}
\usepackage{courier}
% 图形支持宏包
\usepackage{graphicx}
% 插入pdf
\usepackage{pdfpages}
\includepdfset{fitpaper=true}
% 更好的列表环境。
\usepackage{enumitem}       %使用enumitem宏包,改变列表项的格式
\usepackage{enumerate}
\usepackage{environ}
% 禁止 \LaTeX 自动调整多余的页面底部空白,并保持脚注仍然在底部。
% 脚注按页编号。
\usepackage[bottom,perpage,hang]{footmisc}
\raggedbottom
% 脚注格式。
\usepackage{pifont}
% 表格控制
\usepackage{longtable}
\usepackage{booktabs}
% 参考文献引用宏包
\usepackage[sort&compress]{natbib}
% 生成有书签的 pdf 及其开关,请结合 gbk2uni 避免书签乱码。
\usepackage{hyperref}
\hypersetup{
  CJKbookmarks=true,
  linktoc=all,
  bookmarksnumbered=true,
  bookmarksopen=true,
  bookmarksopenlevel=1,
  breaklinks=true,
  colorlinks=false,
  plainpages=false,
  pdfborder=0 0 0}
% 设置 url 样式,与上下文一致
\urlstyle{same}
% 版芯设置
\usepackage{geometry}
\geometry{
  centering,
  text={150true mm,236true mm},
  left=30true mm,
  head=5true mm,
  headsep=2true mm,
  footskip=0true mm,
  foot=5.2true mm
}
% 利用 \pkg{fancyhdr} 设置页眉页脚。
\usepackage{fancyhdr}
% 其他包,表格、数学符号包
\usepackage{tabularx}
\usepackage{varwidth}
% 此处changepage环境用来控制索引页面的左右边距,规范中给出的示例的边距要大于正文。
\usepackage{changepage}
% 多栏结构在文中用begin{multicols}{2}end{multicols}
\usepackage{multicol,multienum}
% 允许上一个section的浮动图形出现在下一个section的开始部分,还提供\FloatBarrier命
% 令,使所有未处理的浮动图形立即被处理
\usepackage[below]{placeins}
% 支持子图 %centerlast 设置最后一行是否居中
\usepackage{subfigure}
% 支持双语标题
\usepackage[subfigure]{ccaption}
% 根据我工规定,正文小四号 (12bp) 字,行距为固定值3--4mm。
\renewcommand\normalsize{%
  % \@setfontsize\normalsize{12bp}{\ifhit@glue 20.50398bp \@plus 2.83465bp \@minus 0bp\else 20.50398bp\fi}%
  \abovedisplayskip=8pt
  \abovedisplayshortskip=8pt
  \belowdisplayskip=\abovedisplayskip
  \belowdisplayshortskip=\abovedisplayshortskip}
% 根据习惯定义字号。用法:\cs{hit@def@fontsize}\marg{字号名称}\marg{磅数}避免了
% 字号选择和行距的紧耦合。所有字号定义时为单倍行距,并提供选项指定行距倍数。
\def\hit@def@fontsize#1#2{%
  \expandafter\newcommand\csname #1\endcsname[1][1.3]{%
    \fontsize{#2}{##1\dimexpr #2}\selectfont}}
\hit@def@fontsize{dachu}{58bp}
\hit@def@fontsize{chuhao}{42bp}
\hit@def@fontsize{xiaochu}{36bp}
\hit@def@fontsize{yihao}{26bp}
\hit@def@fontsize{xiaoyi}{24bp}
\hit@def@fontsize{erhao}{22bp}
\hit@def@fontsize{xiaoer}{18bp}
\hit@def@fontsize{sanhao}{16bp}
\hit@def@fontsize{xiaosan}{15bp}
\hit@def@fontsize{sihao}{14bp}
\hit@def@fontsize{banxiaosi}{13bp}
\hit@def@fontsize{xiaosi}{12bp}
\hit@def@fontsize{dawu}{11bp}
\hit@def@fontsize{wuhao}{10.5bp}
\hit@def@fontsize{xiaowu}{9bp}
\hit@def@fontsize{liuhao}{7.5bp}
\hit@def@fontsize{xiaoliu}{6.5bp}
\hit@def@fontsize{qihao}{5.5bp}
\hit@def@fontsize{bahao}{5bp}
% 利用 \pkg{enumitem} 命令调整默认列表环境间的距离,以符合中文习惯。
\setlist{nosep}
% 允许太长的公式断行、分页等。
\allowdisplaybreaks[4]
\predisplaypenalty=0  %公式之前可以换页,公式出现在页面顶部
\postdisplaypenalty=0
% 公式编号设置
\renewcommand{\theequation}{\arabic{section}.\arabic{equation}}
% 定理标题使用黑体,正文使用宋体,冒号隔开。
\theorembodyfont{\normalfont}
\theoremheaderfont{\normalfont\hei}
\theoremsymbol{\ensuremath{\square}}
\newtheorem*{proof}{证明}
\theoremstyle{plain}
\theoremsymbol{}
\theoremseparator{}
\newtheorem{assumption}{假设}[section]
\newtheorem{definition}{定义}[section]
\newtheorem{proposition}{命题}[section]
\newtheorem{lemma}{引理}[section]
\newtheorem{theorem}{定理}[section]
\newtheorem{axiom}{公理}[section]
\newtheorem{corollary}{推论}[section]
\newtheorem{exercise}{练习}[section]
\newtheorem{example}{例}[section]
\newtheorem{remark}{注释}[section]
\newtheorem{problem}{问题}[section]
\newtheorem{conjecture}{猜想}[section]
% 各种单位
\usepackage{siunitx}
\sisetup{group-minimum-digits=4, group-separator= \hspace{0.25em}}
\sisetup{detect-weight,detect-mode,detect-family}
% 处理数学公式中的黑斜体的宏包
\usepackage{bm}
% 不同于 \mathcal \mathfrak 之类的英文花体字体
\usepackage{mathrsfs}
% 支持彩色
\usepackage{xcolor}
\definecolor{colorzero}{rgb}{0, 0, 0}
\definecolor{colorone}{rgb}{1, 0, 0}
\definecolor{colortwo}{rgb}{0, 0, 1}
\definecolor{colorthree}{rgb}{0, 1, 0}
% 图形和表格的控制旋转
\usepackage{rotating}
% 算法的宏包,注意宏包兼容性,先后顺序为float、hyperref、algorithm(2e),否则无法
% 生成算法列表。
\usepackage[algoruled,linesnumbered]{algorithm2e}
% 排版源码所使用的环境。
\usepackage{listings}
\lstset{
  breaklines  = true,
  captionpos  = b,
  tabsize     = 2,
  numbers     = left,
  columns     = flexible,
  keepspaces  = true,
  % commentstyle = \color[RGB]{0,128,0},
  % keywordstyle = \color[RGB]{0,0,255},
  basicstyle   = \small\ttfamily,
  rulesepcolor = \color{red!20!green!20!blue!20},
  showstringspaces = false,
}

% 作图
\usepackage{tikz}

% 首行缩进
\usepackage{indentfirst}
\setlength{\parindent}{2em}

% 最后定义一些常见的数学公式样式。
\newcommand{\theVector}[1]{\bm{#1}}
\newcommand{\theMatrix}[1]{\mathbb{#1}}
\newcommand{\theSet}[1]{\mathcal{#1}}
\newcommand{\theDirected}[1]{{\overrightarrow{#1}}}
\newcommand{\theUndirected}[1]{{\overline{#1}}}
\newcommand{\theNetwork}[1]{\mathscr{#1}}
\newcommand{\theNode}[1]{{\text{#1}}}
\newcommand{\theDirectedEdge}[2]{{\overrightarrow{{#1}{#2}}}}
\newcommand{\theUndirectedEdge}[2]{{\overline{{#1}{#2}}}}

\graphicspath{{figures/}}

\title{近世代数作业5}
\author{cycleke}

% \renewcommand{\thesection}{}
% \renewcommand{\thesubsection}{}

\begin{document}
\maketitle
\tableofcontents
\clearpage

\section{课后习题}

\subsection{第一题}

\begin{proof*}
  设$U_n = \{x^n = 1| x \in C\}$。

  显然$U_n = \{x_i | 0 \leq i < n, i \in N, x_i = e^\frac{2\pi i}{n} \}$且对于复数的乘法构成群。
  $x_1 = e^\frac{2\pi i}{n}$,而$\forall 1 \leq i \leq n - 1, x_i = x_1^i = e^\frac{2\pi i}{n} \neq 1$,
  而$x_1^n = 1$,所以$|x_1| = n$,$U_n = (x_1)$,即$U_n$为一个循环群。
\end{proof*}

\subsection{第二题}
真子群如下:
\begin{itemize}
\item \{[0]\}
\item \{[0], [6]\}
\item \{[0], [4], [8]\}
\item \{[0], [3], [6], [9]\}
\item \{[0], [2], [4], [6], [8], [10]\}
\end{itemize}

\subsection{第三题}
\begin{proof*}
  首先$\forall x \in (a^r), x = {(a^r)}^k = a^{rk} \in G$,所以$(a^r) \subseteq G$。

  又有因为$G = (a)$,所以$\forall x \in G, x = a^k, 1 \leq k \leq n \text{且} a^n = e$,
  又因为$(r, n) = 1$,所以$\exists p, q \in Z, pr + qn = 1$
  \begin{align*}
    x &= a^k \\
      &= a^{k(pr + qn)} \\
      &= a^{kpr} \circ a^{kqn} \\
      &= {(a^r)}^{kp} \circ {(a^n)}^{kn} \\
      &= {(a^r)}^{kp}
  \end{align*}
  所以$G \subseteq (a^r)$。

  综上所述,$G = (a^r)$。
\end{proof*}

\subsection{第四题}
\begin{proof*}
  因为$a^n = e, (r, n) = d$,所以${(a^r)}^\frac{n}{d} = {(a^n)}^\frac{r}{d} = e$,
  所以$|a^r| \leq \frac{n}{d}$。

  $\forall 1 \leq k < \frac{n}{d}$,若${(a^r)}^k = e$,
  因为$(r, n) = d$,所以$\exists p, q \in Z, pr + qn = d$。
  \begin{align*}
    a^d &= a^{pr + qn} \\
        &= {(a^r)}^p \circ {(a^n)}^q \\
        &= {(a^r)}^p \\
    {(a^d)}^k &= {({(a^r)}^k)}^p \\
        &= e
  \end{align*}
  所以$|a| \leq dk < n$,这与$|a| = n$矛盾。所以$|a^r| \geq \frac{n}{d}$。

  综上所述,$|a^r| = \frac{n}{d}$。
\end{proof*}

\subsection{第五题}
\begin{proof*}
  设$G$是一个六阶群,所以$\forall a \in G, |a| \in \{6, 3, 2, 1\}$。

  若$\exists a, |a| = 6$,则${e, a^2, a^4}$是一个三阶群。

  若$\exists a, |a| = 3$,则${e, a, a^2}$是一个三阶群。

  若$\forall a \in G, |a| \leq 2$,因为有且只有$|e| = 1$,则$\forall a \in G, a^2 = e$。
  由前面的作业有,此时$G$是一个交换群且每个元素的拟元为自身。
  不妨设$a, b \in G, |a| = |b| = 2, a \neq b \neq e$,
  设$S = \{e, a, b, ab\}$,则$\forall x, y \in S, xy^{-1} \in S$,所以$S$为$G$的一个四阶子群,产生矛盾。

  综上所述,六阶群里必有一个三阶子群。
\end{proof*}

\subsection{第六题}
\begin{proof*}
  设$G$是一个$p^m$阶群,$\forall a \in G, |a| = p^k, 0 \leq k <= m$。
  \begin{gather*}
    \therefore a^{p^k} = e \\
    {(a^{p^{k - 1}})}^p = e \\
    \therefore |a^{p^{k - 1}}| \leq p
  \end{gather*}

  因为$p \geq 2$,所以$\exists b \in G, |b^{p^{k - 1}}| = p$,所以存在一个p阶子群$(b^{p^{k - 1}})$.
\end{proof*}

\subsection{第七题}
\begin{align*}
  H &= \{(1\ 2\ 3), (2\ 1\ 3)\} \\
  (3\ 2\ 1)H &= \{(3\ 2\ 1), (3\ 1\ 2)\} \\
  H(3\ 2\ 1) &= \{(3\ 2\ 1), (2\ 3\ 1)\}
\end{align*}

\subsection{第八题}
不一定。

由于$(3\ 2\ 1)$在$S_3$中是一个双射,所以$(3\ 2\ 1)S_3 = S_3(3\ 2\ 1)$。
但是$(3\ 2\ 1)(2\ 1\ 3) = (3\ 1\ 2), (2\ 1\ 3)(3\ 2\ 1) = (2\ 3\ 1)$。

\subsection{第九题}
\begin{proof*}
  设$\varphi: S_l \rightarrow S_r, \forall a\in G, aH \in S_l, \varphi(aH) = Ha^{-1}$。

  首先证明$\varphi$是满射:
  \begin{gather*}
    \forall Hb \in S_r, b, b^{-1} \in G \\
    \exists b^{-1}H \in S_l, \text{即} \varphi(b^{-1}H) = Hb
  \end{gather*}

  再证明$\varphi$是单射:
  \begin{gather*}
    \forall a_1H, a_2H \in S_l, a_1H \neq a_2H \\
    \varphi(a_1H) = Ha_1^{-1} \\
    \varphi(a_2H) = Ha_2^{-1} \\
    \text{若} Ha_1^{-1} = Ha_2^{-1},\text{则} a_1a_2^{-1} \in H \\
    \text{则} a_1H = a_2H, \text{产生矛盾}
  \end{gather*}
  所以$\varphi$是一个单射。

  所以$\varphi$是一个双射,$|S_l| = |S_r|$。
\end{proof*}

\subsection{第十题}
\begin{proof*}
  若$x \in H, x \circ x \in H, x^2 \in H$。

  若$x \notin H, x^{-1} \notin H$,所以$xH \neq H, x^{-1}H \neq H$。
  因为$[G:H]=2$,所以$xH = x^{-1}H, x \circ xH = H, x^2 \in H$。
\end{proof*}
举例:
\begin{itemize}
\item 有限群:G为模n同余类,H为偶余数子群
\item 无限群:G为$(Z, +)$,H为偶数加群
\end{itemize}


\end{document}

%%% Local Variables:
%%% mode: latex
%%% TeX-master: t
%%% End:
