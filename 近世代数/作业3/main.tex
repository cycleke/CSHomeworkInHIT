% !Tex Program = xelatex
% -*-coding: utf-8 -*-
\documentclass[12pt,onecolumn]{article}

% 中文
\usepackage[BoldFont,SlantFont]{xeCJK}
\xeCJKsetemboldenfactor{1}%只对随后定义的CJK字体有效
\setCJKfamilyfont{hei}{SimHei}
\xeCJKsetemboldenfactor{4}
\setCJKfamilyfont{song}{SimSun}
\xeCJKsetemboldenfactor{4}
\setCJKfamilyfont{fs}{FangSong}
\setCJKfamilyfont{kai}{KaiTi}
\setCJKfamilyfont{li}{LiSu}
\setCJKfamilyfont{xw}{STXinwei}
\setCJKmainfont{SimSun}

\newcommand{\hei}{\CJKfamily{hei}}      % 黑体
\newcommand{\song}{\CJKfamily{song}}    % 宋体   (Windows 自带simsun.ttf)
\newcommand{\fs}{\CJKfamily{fs}}        % 仿宋体 (Windows 自带simfs.ttf)
\newcommand{\kai}{\CJKfamily{kai}}      % 楷体   (Windows 自带simkai.ttf)
\newcommand{\li}{\CJKfamily{li}}        % 隶书   (Windows自带simli.ttf)
\newcommand{\xw}{\CJKfamily{xw}}        % 隶书   (Windows自带simli.ttf)

% \AmSTeX\ 宏包,用来排出更加漂亮的公式。
\usepackage{amsmath}
% 定理类环境宏包,其中 \pkg{amsmath} 选项用来兼容 \AmSTeX\ 的宏包
\usepackage[amsmath,thmmarks,hyperref]{ntheorem}
\usepackage{amssymb}
% 添加字体
\usepackage[defaultsups]{newtxtext}
\usepackage{newtxmath}
\usepackage{courier}
% 图形支持宏包
\usepackage{graphicx}
% 插入pdf
\usepackage{pdfpages}
\includepdfset{fitpaper=true}
% 更好的列表环境。
\usepackage{enumitem}       %使用enumitem宏包,改变列表项的格式
\usepackage{enumerate}
\usepackage{environ}
% 禁止 \LaTeX 自动调整多余的页面底部空白,并保持脚注仍然在底部。
% 脚注按页编号。
\usepackage[bottom,perpage,hang]{footmisc}
\raggedbottom
% 脚注格式。
\usepackage{pifont}
% 表格控制
\usepackage{longtable}
\usepackage{booktabs}
% 参考文献引用宏包
\usepackage[sort&compress]{natbib}
% 生成有书签的 pdf 及其开关,请结合 gbk2uni 避免书签乱码。
\usepackage{hyperref}
\hypersetup{
  CJKbookmarks=true,
  linktoc=all,
  bookmarksnumbered=true,
  bookmarksopen=true,
  bookmarksopenlevel=1,
  breaklinks=true,
  colorlinks=false,
  plainpages=false,
  pdfborder=0 0 0}
% 设置 url 样式,与上下文一致
\urlstyle{same}
% 版芯设置
\usepackage{geometry}
\geometry{
  centering,
  text={150true mm,236true mm},
  left=30true mm,
  head=5true mm,
  headsep=2true mm,
  footskip=0true mm,
  foot=5.2true mm
}
% 利用 \pkg{fancyhdr} 设置页眉页脚。
\usepackage{fancyhdr}
% 其他包,表格、数学符号包
\usepackage{tabularx}
\usepackage{varwidth}
% 此处changepage环境用来控制索引页面的左右边距,规范中给出的示例的边距要大于正文。
\usepackage{changepage}
% 多栏结构在文中用begin{multicols}{2}end{multicols}
\usepackage{multicol,multienum}
% 允许上一个section的浮动图形出现在下一个section的开始部分,还提供\FloatBarrier命
% 令,使所有未处理的浮动图形立即被处理
\usepackage[below]{placeins}
% 支持子图 %centerlast 设置最后一行是否居中
\usepackage{subfigure}
% 支持双语标题
\usepackage[subfigure]{ccaption}
% 根据我工规定,正文小四号 (12bp) 字,行距为固定值3--4mm。
\renewcommand\normalsize{%
  % \@setfontsize\normalsize{12bp}{\ifhit@glue 20.50398bp \@plus 2.83465bp \@minus 0bp\else 20.50398bp\fi}%
  \abovedisplayskip=8pt
  \abovedisplayshortskip=8pt
  \belowdisplayskip=\abovedisplayskip
  \belowdisplayshortskip=\abovedisplayshortskip}
% 根据习惯定义字号。用法:\cs{hit@def@fontsize}\marg{字号名称}\marg{磅数}避免了
% 字号选择和行距的紧耦合。所有字号定义时为单倍行距,并提供选项指定行距倍数。
\def\hit@def@fontsize#1#2{%
  \expandafter\newcommand\csname #1\endcsname[1][1.3]{%
    \fontsize{#2}{##1\dimexpr #2}\selectfont}}
\hit@def@fontsize{dachu}{58bp}
\hit@def@fontsize{chuhao}{42bp}
\hit@def@fontsize{xiaochu}{36bp}
\hit@def@fontsize{yihao}{26bp}
\hit@def@fontsize{xiaoyi}{24bp}
\hit@def@fontsize{erhao}{22bp}
\hit@def@fontsize{xiaoer}{18bp}
\hit@def@fontsize{sanhao}{16bp}
\hit@def@fontsize{xiaosan}{15bp}
\hit@def@fontsize{sihao}{14bp}
\hit@def@fontsize{banxiaosi}{13bp}
\hit@def@fontsize{xiaosi}{12bp}
\hit@def@fontsize{dawu}{11bp}
\hit@def@fontsize{wuhao}{10.5bp}
\hit@def@fontsize{xiaowu}{9bp}
\hit@def@fontsize{liuhao}{7.5bp}
\hit@def@fontsize{xiaoliu}{6.5bp}
\hit@def@fontsize{qihao}{5.5bp}
\hit@def@fontsize{bahao}{5bp}
% 利用 \pkg{enumitem} 命令调整默认列表环境间的距离,以符合中文习惯。
\setlist{nosep}
% 允许太长的公式断行、分页等。
\allowdisplaybreaks[4]
\predisplaypenalty=0  %公式之前可以换页,公式出现在页面顶部
\postdisplaypenalty=0
% 公式编号设置
\renewcommand{\theequation}{\arabic{section}.\arabic{equation}}
% 定理标题使用黑体,正文使用宋体,冒号隔开。
\theorembodyfont{\normalfont}
\theoremheaderfont{\normalfont\hei}
\theoremsymbol{\ensuremath{\square}}
\newtheorem*{proof}{证明}
\theoremstyle{plain}
\theoremsymbol{}
\theoremseparator{}
\newtheorem{assumption}{假设}[section]
\newtheorem{definition}{定义}[section]
\newtheorem{proposition}{命题}[section]
\newtheorem{lemma}{引理}[section]
\newtheorem{theorem}{定理}[section]
\newtheorem{axiom}{公理}[section]
\newtheorem{corollary}{推论}[section]
\newtheorem{exercise}{练习}[section]
\newtheorem{example}{例}[section]
\newtheorem{remark}{注释}[section]
\newtheorem{problem}{问题}[section]
\newtheorem{conjecture}{猜想}[section]
% 各种单位
\usepackage{siunitx}
\sisetup{group-minimum-digits=4, group-separator= \hspace{0.25em}}
\sisetup{detect-weight,detect-mode,detect-family}
% 处理数学公式中的黑斜体的宏包
\usepackage{bm}
% 不同于 \mathcal \mathfrak 之类的英文花体字体
\usepackage{mathrsfs}
% 支持彩色
\usepackage{xcolor}
\definecolor{colorzero}{rgb}{0, 0, 0}
\definecolor{colorone}{rgb}{1, 0, 0}
\definecolor{colortwo}{rgb}{0, 0, 1}
\definecolor{colorthree}{rgb}{0, 1, 0}
% 图形和表格的控制旋转
\usepackage{rotating}
% 算法的宏包,注意宏包兼容性,先后顺序为float、hyperref、algorithm(2e),否则无法
% 生成算法列表。
\usepackage[algoruled,linesnumbered]{algorithm2e}
% 排版源码所使用的环境。
\usepackage{listings}
\lstset{
  breaklines  = true,
  captionpos  = b,
  tabsize     = 2,
  numbers     = left,
  columns     = flexible,
  keepspaces  = true,
  % commentstyle = \color[RGB]{0,128,0},
  % keywordstyle = \color[RGB]{0,0,255},
  basicstyle   = \small\ttfamily,
  rulesepcolor = \color{red!20!green!20!blue!20},
  showstringspaces = false,
}

% 作图
\usepackage{tikz}

% 首行缩进
\usepackage{indentfirst}
\setlength{\parindent}{2em}

% 最后定义一些常见的数学公式样式。
\newcommand{\theVector}[1]{\bm{#1}}
\newcommand{\theMatrix}[1]{\mathbb{#1}}
\newcommand{\theSet}[1]{\mathcal{#1}}
\newcommand{\theDirected}[1]{{\overrightarrow{#1}}}
\newcommand{\theUndirected}[1]{{\overline{#1}}}
\newcommand{\theNetwork}[1]{\mathscr{#1}}
\newcommand{\theNode}[1]{{\text{#1}}}
\newcommand{\theDirectedEdge}[2]{{\overrightarrow{{#1}{#2}}}}
\newcommand{\theUndirectedEdge}[2]{{\overline{{#1}{#2}}}}

\graphicspath{{figures/}}

\title{近世代数作业3}
\author{cycleke}

% \renewcommand{\thesection}{}
% \renewcommand{\thesubsection}{}

\begin{document}
\maketitle
\tableofcontents
\clearpage

\section{课后习题}

\subsection{第一题}
\begin{proof*}
  首先证明$(S, \circ)$是一个代数系,即$\circ$在$S$上封闭。

  \begin{gather*}
    \forall (a,b), (c, d) \in S \\
    (a, b) \circ (c, d) = (ac, ad + b) \\
    \because a, c \neq 0 , a, b, c, d \in R \\
    \therefore ac \neq 0, ac, ad + b \in R \\
    \therefore (ac, ad + b) \in S
  \end{gather*}

  再证明$(S, \circ)$是一个半群。

  \begin{align*}
    \forall (a, b), (c, d), (e, f) &\in S \\
    ((a, b) \circ (c, d)) \circ (e, f) &= (ac, ad + b) \circ (e, f) \\
                             &= (ace, acf + ad + b) \\
    (a, b) \circ ((c, d) \circ (e, f)) &= (a, b) \circ (ce, cf + d) \\
                             &= (ace, acf + ad + b) \\
    \therefore ((a, b) \circ (c, d)) \circ (e, f) &= (a, b) \circ ((c, d) \circ (e, f))
  \end{align*}

  又有
  \[
    \forall (a, b) \in S,
    (1, 0) \circ (a, b) = (a, b),
    (\frac{1}{a}, -\frac{b}{a}) \circ (a, b) = (1, 0)
  \]

  所以$(S, \circ)$是群。
\end{proof*}

\subsection{第二题}
\begin{proof*}
  \begin{gather*}
    \because {(ab)}^2 = a^2b^2 \\
    \therefore abab = aabb \\
    \Rightarrow a^{-1}abab = a^{-1}aabb
    \Rightarrow bab = abb \\
    \Rightarrow babb^{-1} = abbb^{-1}
    \Rightarrow ba = ab \\
    \therefore ab = ba
  \end{gather*}
\end{proof*}

\subsection{第三题}
\begin{proof*}
  因为$\forall a \in G, a^2 = e$,又因为$G$是群,所以$a^{-1} = a$。
  \begin{align*}
    \forall a, b \in G, ab &= {(ab)}^{-1} \\
                   &= b^{-1}a^{-1} \\
                   &= {(b^{-1})}^{-1}{(a^{-1})}^{-1} \\
                   &= ba
  \end{align*}

  所以$G$是交换群。
\end{proof*}

\subsection{第四题}
\begin{proof*}
  设$G$是四阶群,则设$G = \{e, a, b, c\}$。
  若$G$不是交换群,不妨$ab \neq ba$。
  显然$ab \neq a, ab \neq b, ba \neq a, ba \neq b$,所以可以设$ab = c, ba = e$。
  因为$G$是一个群,进而有$b = a^{-1} \Rightarrow ab = e$,这与$ab = c$矛盾,所以$ab = ba$。
  同理可以推出其它的交换成立,所以$G$是一个交换群。
\end{proof*}

\subsection{第五题}
\begin{proof*}
  设该群为$G$,若$\exists a \in G,|a| \geq 3$,则$a \neq a^{-1}(\text{否则}a^2 = e \Rightarrow |a| = 2)$。
  所以$aa^{-1} = a^{-1}a = e$。

  若$\nexists a \in G, |a| \geq 3$,即$\forall a \in G, |a| \leq 2 \Rightarrow \forall a \in G, a^2 = e$。
  由第三题的证明可知,可以证明$G$是一个交换群,这与题设矛盾。

  综上所述,命题得证。
\end{proof*}

\subsection{第六题}
\begin{proof*}
  设$G$是一个有限群,对于所有的$\forall a \in G, s.t. |a| = 2$
  有$a^2 = e \iff a = a^{-1}$。

  由于$a$与$a^{-1}$同阶,所以$\forall a\in G, s.t. |2| > a$
  有$|a| \neq |a^{-1}|, |a^{-1}| = |a| > 2$,
  所以有限群里阶大于 2 的元素的个数必为偶数。
\end{proof*}

\subsection{第七题}
\begin{proof*}
  由第六题知,可以知在偶数阶群中,阶大于2的元素为偶数个,
  所以阶小于等于2的元素为偶数个。
  又有$|e| = 1$,所以偶数阶群里阶为2的元素的个数必为奇数。
\end{proof*}

\subsection{第八题}
\begin{proof*}
  由第七题知,可以知在偶数阶群中,偶数阶群里阶为 2 的元素的个数必为奇数。
  有因为偶数阶群里阶为 2 的元素的个数非负,所以偶数阶群里至少有一个阶为2的元素。
\end{proof*}

\subsection{第九题}
\begin{proof*}
  设$b_0 = e, b_i = \prod_{j=1}^{i}a_j, i = 1, 2, \ldots, n$,
  由抽屉原理有$\exists i, j \in {0, 1, \ldots, n}, i < j, b_i = b_j$。

  \begin{align*}
    e \circ a_1 \circ a_2 \circ \ldots \circ a_i &= e \circ a_1 \circ a_2 \circ \ldots \circ a_j \\
    a_1 \circ a_2 \circ \ldots \circ a_i &= a_1 \circ a_2 \circ \ldots \circ a_j \\
    a_2 \circ \ldots \circ a_i &= a_2 \circ \ldots \circ a_j \\
                            &\cdots \\
    e &= a_{i + 1} \circ a_{i + 2} \circ \ldots \circ a_j \\
  \end{align*}

  所以$ \left\{
    \begin{array}{l}
      p = i + 1 \\
      q = j
    \end{array} \right.$。
\end{proof*}

\subsection{第十题}
\begin{proof*}
  设$l = [m, n], |ab| = k$,则因为$ab = ba$

  \begin{align*}
    {(ab)}^l &= a^l b^l \\
             &= {a^m}^{\frac{l}{m}}{b^n}^{\frac{l}{n}} \\
             &= e
  \end{align*}

  有因为$|ab| = k$,所以$k \leq l$,
  若$k \nmid l$,则$l = pk + q, p, q \in \theSet{N^{+}}, 1 \leq q < k$。

  \begin{align*}
    {(ab)}^l &= a^l b^l \\
             &= a^{pk} a^q b^{pk} b^{q} \\
             &= a^q b^q \\
             &= {(ab)}^q \neq e
  \end{align*}

  发生矛盾,所以$k \mid l$。
\end{proof*}

当$(n, m) = 1$时,$ab$的阶为$mn$。

\end{document}

%%% Local Variables:
%%% mode: latex
%%% TeX-master: t
%%% End:
